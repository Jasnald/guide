\chapter{Experimental Data Processing}
\label{ch:exp_process}

\textbf{Module Path:} \texttt{src/exp\_process/}

This module implements the full preprocessing and fitting pipeline for raw experimental surface and curve measurements. It replaces both the previous \texttt{Exp\_Data} and \texttt{Preprocess} modules with a more structured, object-oriented architecture designed to be extended for new experiment types with minimal changes to existing code.

Previously, \texttt{Exp\_Data} handled physical dimensions and sample configuration while \texttt{Preprocess} ran separate per-experiment scripts for cleaning, fitting, and visualization. These responsibilities are now unified under \texttt{exp\_process}, with the processing logic reorganized into reusable, experiment-agnostic layers. Sample dimensions and simulation configuration remain in separate modules outside \texttt{exp\_process} (see \texttt{data/input/}).

The module handles two distinct experiment types, each with their own geometry and data format:

\begin{itemize}
    \item \textbf{Experiment 1 (T-Shape):} 3D surface measurements from a T-shaped specimen, acquired in multiple passes (\texttt{bottom} and \texttt{wall} regions) across two measurement sides. Multiple measurements per side are averaged during parsing.
    \item \textbf{Experiment 2 (Rectangular Profile):} 1D curve measurements from a rectangular specimen, acquired in two opposing directions (Left/Right). The L/R pair is averaged and merged into a single profile during parsing, before any further processing.
\end{itemize}

\section{Module Architecture}

The module is organized into five layers, each with a well-defined responsibility. Data flows top-to-bottom through these layers, with configuration injected at the pipeline level.

\begin{enumerate}
    \item \textbf{Parsers (\texttt{parsers/}):} Read raw \texttt{.txt} files from disk and return structured NumPy arrays. One parser class per experiment type.
    \item \textbf{Core (\texttt{core/}):} Stateless mathematical and geometric utilities — cleaning, fitting, meshing, transforming, segmenting, and reconstructing. These classes have no knowledge of file paths or experiment types.
    \item \textbf{Procedures (\texttt{procedures/}):} Orchestrate the core utilities into named pipeline stages: \texttt{preprocess}, \texttt{fitting}, \texttt{comparison}, and \texttt{validation}. Each procedure reads from and writes to disk as JSON.
    \item \textbf{Pipeline (\texttt{pipeline/}):} High-level entry points that chain procedures together. A user runs the pipeline; the pipeline calls the procedures in order.
    \item \textbf{Utils \& GUI (\texttt{utils/}, \texttt{gui/}):} JSON I/O helpers and the interactive point cloud viewer used in the validation step.
\end{enumerate}

\begin{figure}[ht]
    \centering
    \begin{verbatim}
    [Raw .txt files]
         |
    [ Parsers ]  <-- reading + averaging (Exp1: multi-pass; Exp2: L/R merge)
         |
    [ Procedures: Preprocess ]  <-- IQR cleaning, step segmentation, JSON output
         |
    [ GUI: Validation ]  <-- manual point deletion (blocking)
         |
    [ Procedures: Fitting ]  <-- 1D/2D polynomial fitting, JSON output
         |
    [ Procedures: Comparison ]  <-- subtraction, averaging, rebuild output
    \end{verbatim}
    \caption{Data flow through the \texttt{exp\_process} pipeline.}
    \label{fig:exp_process_flow}
\end{figure}

\section{Dependencies}

The module depends on standard scientific Python libraries. All are listed in \texttt{src/exp\_process/deps\_core.py} and \texttt{deps\_gui.py}:

\begin{itemize}
    \item \textbf{Core:} \texttt{numpy}, \texttt{shapely}, \texttt{scipy} (implicit via \texttt{numpy.linalg})
    \item \textbf{GUI:} \texttt{matplotlib} (TkAgg backend), \texttt{tkinter}
\end{itemize}

\section{How to Run}

calma calma calma calma calma calma


\section{File Structure}

\begin{verbatim}
src/exp_process/
    deps_core.py          # core imports
    deps_gui.py           # GUI imports
    parsers/
        _base.py          # AbstractParser
        t_shape.py        # Experiment 1 parser
        rec_shape.py      # Experiment 2 parser
    core/
        cleaner.py        # IQR outlier removal
        fitter.py         # 1D and 2D polynomial fitting
        mesher.py         # grid generation
        operations.py     # model arithmetic (subtract, average)
        rebuilder.py      # surface/curve reconstruction
        segmenter.py      # step detection
        transformer.py    # geometric corrections
    procedures/
        preprocess.py     # Stage 1: clean and segment
        fitting.py        # Stage 3: polynomial fitting
        comparison.py     # Stage 4: model averaging
        validation.py     # Stage 2: GUI launcher
    pipeline/
        base.py           # BasePipeline (abstract)
        surface.py        # Experiment 1 full pipeline
        curve.py          # Experiment 2 full pipeline
    gui/
        viewer.py         # PointCloudViewer
    utils/
        io.py             # JSON save/load utilities
\end{verbatim}

calma calma calma calma calma calma

\section{Chapter Organization}

The following sections document each layer in detail, including configuration parameters, extension points, and common modification scenarios.

\section{Parsers}
\label{sec:parsers}

\textbf{Module path:} \texttt{src/exp\_process/parsers/}

Parsers are responsible for reading raw \texttt{.txt} files from disk and returning structured NumPy arrays. They are the only layer with knowledge of the input directory structure and file naming conventions. All downstream code receives plain arrays and has no dependency on file paths.

Each parser implements the \texttt{AbstractParser} interface and handles one experiment type. The output of \texttt{load()} is always a \texttt{dict} mapping an identifier string to a NumPy array, so all procedures can iterate over them uniformly regardless of experiment type.

\subsection{AbstractParser}
\label{subsec:abstract_parser}

\textbf{File:} \texttt{parsers/\_base.py}

Defines the interface that all parsers must implement. Contains no logic.

\begin{table}[ht]
\centering
\begin{tabular}{lll}
\hline
\textbf{Method} & \textbf{Signature} & \textbf{Description} \\
\hline
\texttt{\_\_init\_\_} & \texttt{(input\_dir: str)} & Stores the input directory path. \\
\texttt{load} & \texttt{(target\_id: str) -> dict} & Load raw data for a given ID. \\
\texttt{list\_ids} & \texttt{() -> list} & List all available IDs in \texttt{input\_dir}. \\
\hline
\end{tabular}
\caption{AbstractParser interface.}
\label{tab:abstract_parser}
\end{table}

To add a new experiment type, subclass \texttt{AbstractParser} and implement both methods. No changes to procedures or pipeline are required as long as the return format is respected.

\subsection{TShapeParser — Experiment 1}
\label{subsec:tshape_parser}

\textbf{File:} \texttt{parsers/t\_shape.py}

Reads surface measurement data from T-shaped specimens. The input directory is expected to contain files named:

\begin{verbatim}
{SideID}_Measurment{N}_bottom.txt
{SideID}_Measurment{N}_wall.txt
\end{verbatim}

where \texttt{N} is the measurement number (integer) and \texttt{SideID} is typically \texttt{Side1} or \texttt{Side2}. Each side has multiple measurements (repetitions), and each measurement is split into a \texttt{bottom} region and a \texttt{wall} region in separate files.

\textbf{load(side\_id)} stacks \texttt{bottom} and \texttt{wall} arrays vertically per measurement using \texttt{np.vstack}, returning:

\begin{lstlisting}[language=Python]
{
    "1": np.ndarray,  # measurement 1: bottom + wall
    "2": np.ndarray,  # measurement 2: bottom + wall
    ...
}
\end{lstlisting}

The merge of \texttt{bottom} and \texttt{wall} preserves region identity only implicitly through row ordering. If a region file is missing for a given measurement number, that measurement is skipped with a printed warning.

\textbf{list\_ids()} scans \texttt{input\_dir} for filenames matching the naming pattern and returns the unique side identifiers found (e.g., \texttt{['Side1', 'Side2']}).

\subsection{RecShapeParser — Experiment 2}
\label{subsec:recshape_parser}

\textbf{File:} \texttt{parsers/rec\_shape.py}

Reads 1D curve measurement data from rectangular specimens. The input directory is expected to contain subfolders named:

\begin{verbatim}
{sample_id}L/   <-- contains one .txt or .csv file
{sample_id}R/   <-- contains one .txt or .csv file
\end{verbatim}

where \texttt{sample\_id} is a numeric string (e.g., \texttt{1}, \texttt{2}). L and R represent measurements taken from opposing directions on the same specimen, forming a single geometric profile when combined.

\textbf{load(sample\_id)} merges L and R as follows: R is reversed (\texttt{[::-1]}) to align acquisition direction with L, both are truncated to the shorter length, and their element-wise average is computed. The result is a single array representing the merged profile:

\begin{lstlisting}[language=Python]
{"1": np.ndarray}  # averaged L/R profile for sample 1
\end{lstlisting}

If either folder or file is missing, the sample is skipped with a printed warning.

\textbf{list\_ids()} scans \texttt{input\_dir} for folders matching \texttt{\^{}(\textbackslash d+)[LR]\$} and returns unique numeric identifiers (e.g., \texttt{['1', '2', '3']}).

\subsection{Adding a New Parser}

To support a new experiment type:

\begin{enumerate}
    \item Create \texttt{parsers/my\_type.py} with a class inheriting \texttt{AbstractParser}.
    \item Implement \texttt{list\_ids()} to scan the input directory.
    \item Implement \texttt{load(target\_id)} to return \texttt{\{id: np.ndarray\}}.
    \item Create the corresponding pipeline entry point in \texttt{pipeline/}.
\end{enumerate}

No changes to \texttt{core/} or \texttt{procedures/} are required.
\section{Core}
\label{sec:core}

\textbf{Module path:} \texttt{src/exp\_process/core/}

The core layer contains stateless utilities for all mathematical and geometric operations. No class here reads from or writes to disk, and none has knowledge of experiment types. All methods are \texttt{@staticmethod} (except \texttt{DataTransformer}, which holds configuration state). This design allows procedures to compose core utilities freely without coupling.

\subsection{OutlierCleaner}
\label{subsec:cleaner}

\textbf{File:} \texttt{core/cleaner.py}

Removes outliers from point arrays using the Interquartile Range (IQR) method. Each axis is filtered independently using the configured factor.

\begin{table}[ht]
\centering
\begin{tabular}{lp{9cm}}
\hline
\textbf{Method} & \textbf{Description} \\
\hline
\texttt{filter\_iqr(data, factors)} & Removes points outside $[Q1 - f \cdot IQR,\ Q3 + f \cdot IQR]$ for each axis specified in \texttt{factors}. Returns the filtered array. \\
\hline
\end{tabular}
\caption{OutlierCleaner interface.}
\end{table}

\texttt{data} must be shape \texttt{(N, 3)}. \texttt{factors} is a dict mapping axis names to multipliers, e.g.:

\begin{lstlisting}[language=Python]
OutlierCleaner.filter_iqr(points, {'x': 1.5, 'y': 1.5, 'z': 1.2})
\end{lstlisting}

Axes not present in \texttt{factors} are not filtered. If \texttt{data} is empty, it is returned unchanged.

\subsection{Fitter}
\label{subsec:fitter}

\textbf{File:} \texttt{core/fitter.py}

Polynomial fitting and evaluation for both 1D (Exp2 curves) and 2D (Exp1 surfaces). All methods return a model \texttt{dict} that encodes the polynomial type, degree, and coefficients, allowing evaluation to be decoupled from fitting.

\begin{table}[ht]
\centering
\begin{tabular}{lp{9cm}}
\hline
\textbf{Method} & \textbf{Description} \\
\hline
\texttt{fit\_1d\_poly(x, z, degree, ...)} & Fits a 1D polynomial. Supports optional x-normalization to $[-1, 1]$ and Ridge regularization ($\lambda > 0$). Returns coefficients back-transformed to the original x scale. \\
\texttt{eval\_1d\_poly(x, model)} & Evaluates a 1D model at scalar or array \texttt{x} using \texttt{np.polyval}. \\
\texttt{fit\_2d\_poly(x, y, z, degree)} & Fits a separable 2D polynomial of the form $z = \sum_{k=1}^{d} (a_k x^k + b_k y^k) + c$. Uses \texttt{np.linalg.lstsq}. \\
\texttt{eval\_2d\_poly(x, y, model)} & Evaluates a 2D model at arrays \texttt{x}, \texttt{y}. Returns a \texttt{np.ndarray}. \\
\hline
\end{tabular}
\caption{Fitter interface.}
\end{table}

The 2D polynomial is separable: cross-terms ($x^i y^j$, $i,j > 0$) are not included. The coefficient vector layout is \texttt{[a\_1, b\_1, a\_2, b\_2, ..., c]} where \texttt{c} is the constant bias stored last. This layout is shared with \texttt{ModelOps}.

\textbf{Model dict structure:}

\begin{lstlisting}[language=Python]
# 1D
{"type": "poly_1d", "degree": 4, "coeffs": [...], "norm": {...}, "fit": {...}}

# 2D
{"type": "poly_2d", "degree": 6, "coeffs": [...]}
\end{lstlisting}

\subsection{MeshGenerator}
\label{subsec:mesher}

\textbf{File:} \texttt{core/mesher.py}

Generates structured point grids for surface reconstruction. Used by \texttt{Rebuilder}.

\begin{table}[ht]
\centering
\begin{tabular}{lp{9cm}}
\hline
\textbf{Method} & \textbf{Description} \\
\hline
\texttt{rectangular\_grid(width, height, step)} & Regular grid over $[0, \text{width}] \times [0, \text{height}]$ with spacing \texttt{step}. Returns flat \texttt{(x, y)} arrays. \\
\texttt{t\_shape\_grid(dims, step)} & Grid of points inside a T-shaped polygon, built as the union of two rectangles (horizontal flange and vertical web). Uses Shapely for containment testing. Returns flat \texttt{(x, y)} arrays. \\
\hline
\end{tabular}
\caption{MeshGenerator interface.}
\end{table}

\texttt{t\_shape\_grid} expects a \texttt{dims} dict with keys \texttt{h\_width}, \texttt{h\_thickness}, \texttt{v\_width}, \texttt{v\_height}, and optionally \texttt{offset\_1} for the horizontal position of the web. If \texttt{offset\_1} is absent, the web is centered on the flange.

\subsection{ModelOps}
\label{subsec:operations}

\textbf{File:} \texttt{core/operations.py}

Arithmetic operations on fitted model dicts. Handles coefficient alignment between models of different degrees, with separate logic for 1D and 2D coefficient layouts.

\begin{table}[ht]
\centering
\begin{tabular}{lp{9cm}}
\hline
\textbf{Method} & \textbf{Description} \\
\hline
\texttt{subtract\_coeffs(model\_high, model\_low)} & Subtracts \texttt{model\_low} from \texttt{model\_high}, padding the lower-degree model with zeros as needed. For 1D, pads from the left (high-degree terms); for 2D, pads term-by-term preserving the bias. \\
\texttt{average\_models(models)} & Computes the element-wise mean of coefficients across a list of same-degree models. Raises \texttt{ValueError} if degrees differ. \\
\hline
\end{tabular}
\caption{ModelOps interface.}
\end{table}

\texttt{model\_high} must have degree $\geq$ \texttt{model\_low}. The returned dict preserves \texttt{type}, \texttt{degree}, and \texttt{norm} from \texttt{model\_high}.

\subsection{Rebuilder}
\label{subsec:rebuilder}

\textbf{File:} \texttt{core/rebuilder.py}

Reconstructs point clouds from fitted models by evaluating them on structured grids. Composes \texttt{MeshGenerator} and \texttt{Fitter}.

\begin{table}[ht]
\centering
\begin{tabular}{lp{9cm}}
\hline
\textbf{Method} & \textbf{Description} \\
\hline
\texttt{rebuild\_surface(model, geometry\_type, dims, step)} & Evaluates a 2D model on a \texttt{t\_shape} or \texttt{rectangular} grid. Returns an \texttt{(N, 3)} array. \\
\texttt{rebuild\_curve\_extrusion(model\_1d, dims, step\_x, step\_y)} & Evaluates a 1D model along x and extrudes uniformly in y. Returns an \texttt{(N, 3)} array. \\
\hline
\end{tabular}
\caption{Rebuilder interface.}
\end{table}

\subsection{StepSegmenter}
\label{subsec:segmenter}

\textbf{File:} \texttt{core/segmenter.py}

Detects step discontinuities in point clouds by identifying large relative jumps in x-values after sorting. Used in the Exp1 preprocessing stage to split the merged bottom+wall cloud into individual measurement steps.

\begin{table}[ht]
\centering
\begin{tabular}{lp{9cm}}
\hline
\textbf{Method} & \textbf{Description} \\
\hline
\texttt{find\_steps(points, threshold\_percent)} & Sorts points by \texttt{(x, y)}, computes relative x-differences, and splits at indices where the difference exceeds \texttt{threshold\_percent} (default \texttt{0.6\%}). Returns a list of arrays. \\
\hline
\end{tabular}
\caption{StepSegmenter interface.}
\end{table}

\subsection{DataTransformer}
\label{subsec:transformer}

\textbf{File:} \texttt{core/transformer.py}

Applies per-side geometric corrections (mirroring, inversion, coordinate offsets) to raw point arrays before fitting. Unlike the other core classes, \texttt{DataTransformer} is instantiated with a \texttt{rules} dict that maps side IDs to their transformations.

\begin{lstlisting}[language=Python]
transformer = DataTransformer(rules={
    "Side2": {"mirror_x": True, "invert_z": True},
    "Side1": {"offset_x": 5.0}
})
corrected = transformer.apply("Side2", points)
\end{lstlisting}

\begin{table}[ht]
\centering
\begin{tabular}{lp{8cm}}
\hline
\textbf{Rule key} & \textbf{Effect} \\
\hline
\texttt{mirror\_x} & Reflects x around \texttt{mirror\_ref} (defaults to \texttt{max(x)}). \\
\texttt{invert\_z} & Negates the z (or y for 2-column arrays) coordinate. \\
\texttt{offset\_x/y/z} & Adds a scalar offset to the respective coordinate. \\
\hline
\end{tabular}
\caption{DataTransformer rule keys.}
\end{table}

Sides with no entry in \texttt{rules} are returned unchanged. Points must have shape \texttt{(N, 2)} or \texttt{(N, 3)}.
\section{Procedures}
\label{sec:procedures}

\textbf{Module path:} \texttt{src/exp\_process/procedures/}

Procedures are the pipeline stages. Each procedure orchestrates core utilities into a named, configurable step: it reads input files, calls core methods, and writes JSON output. Procedures know about experiment types and file paths; core classes do not.

All procedures follow the same pattern: receive a config dataclass, return a \texttt{\{id: Path\}} dict mapping each processed item to its output file. An empty dict means nothing was processed (missing files or empty data).

\subsection{Configuration Dataclasses}
\label{subsec:proc_configs}

\textbf{File:} \texttt{procedures/preprocess.py}

\texttt{BasePreprocessConfig} is the shared base for all configs. It holds \texttt{input\_dir} and \texttt{output\_dir} as \texttt{Path} objects and creates \texttt{output\_dir} on instantiation.

\begin{table}[ht]
\centering
\begin{tabular}{llll}
\hline
\textbf{Config} & \textbf{Field} & \textbf{Default} & \textbf{Description} \\
\hline
\multirow{4}{*}{\texttt{Exp1PreprocessConfig}}
  & \texttt{outlier\_bottom} & \texttt{1.2} & IQR factor for bottom region. \\
  & \texttt{outlier\_top}    & \texttt{1.2} & IQR factor for wall region. \\
  & \texttt{outlier\_general}& \texttt{1.5} & IQR factor for final merged cloud. \\
  & \texttt{step\_threshold} & \texttt{0.6} & Relative x-jump threshold (\%) for step detection. \\
\hline
\multirow{2}{*}{\texttt{Exp2PreprocessConfig}}
  & \texttt{x\_col} & \texttt{0} & Column index for X in raw data array. \\
  & \texttt{z\_col} & \texttt{1} & Column index for Z in raw data array. \\
\hline
\end{tabular}
\caption{Preprocessing config fields.}
\end{table}

\textbf{File:} \texttt{procedures/fitting.py}

\begin{table}[ht]
\centering
\begin{tabular}{llll}
\hline
\textbf{Config} & \textbf{Field} & \textbf{Default} & \textbf{Description} \\
\hline
\multirow{2}{*}{\texttt{Exp1FittingConfig}}
  & \texttt{high\_degree} & \texttt{4} & Degree for the detailed surface fit. \\
  & \texttt{fix\_rules}   & \texttt{None} & Transformation rules for \texttt{DataTransformer}. \\
\hline
\multirow{4}{*}{\texttt{Exp2FittingConfig}}
  & \texttt{high\_degree}   & \texttt{2}    & Degree for the detailed curve fit. \\
  & \texttt{normalize\_x}   & \texttt{True} & Normalize x to $[-1,1]$ before fitting. \\
  & \texttt{ridge\_alpha}   & \texttt{1.0}  & Ridge regularization factor ($\lambda$). \\
  & \texttt{fix\_rules}     & \texttt{None} & Transformation rules for \texttt{DataTransformer}. \\
\hline
\end{tabular}
\caption{Fitting config fields.}
\end{table}

\subsection{preprocess\_exp1}
\label{subsec:preprocess_exp1}

\textbf{File:} \texttt{procedures/preprocess.py}

Processes all sides found in \texttt{input\_dir} through the following sequence:

\begin{enumerate}
    \item \texttt{TShapeParser.load(side\_id)} — loads and stacks bottom+wall measurements for all repetitions.
    \item \texttt{np.vstack} — merges all repetitions into a single point cloud per side.
    \item \texttt{OutlierCleaner.filter\_iqr} — removes outliers on the z axis using \texttt{outlier\_general}.
    \item \texttt{StepSegmenter.find\_steps} — splits the cloud into measurement steps.
    \item \texttt{IOUtils.save\_result} — writes \texttt{\{side\_id\}\_Steps.json}.
\end{enumerate}

Output structure per side:
\begin{lstlisting}[language=Python]
{
    "id": "Side1",
    "total_steps": 12,
    "steps": [
        {"step_number": 1, "point_count": 84, "points": [...]},
        ...
    ]
}
\end{lstlisting}

\subsection{preprocess\_exp2}
\label{subsec:preprocess_exp2}

\textbf{File:} \texttt{procedures/preprocess.py}

Processes all samples found in \texttt{input\_dir}:

\begin{enumerate}
    \item \texttt{RecShapeParser.load(sample\_id)} — loads and averages L/R files.
    \item \texttt{IOUtils.save\_result} — writes \texttt{\{sample\_id\}\_Raw.json}.
\end{enumerate}

No outlier removal or segmentation occurs here for Exp2. The raw merged profile is saved as-is for manual inspection via the GUI before fitting.

\subsection{fit\_exp1}
\label{subsec:fit_exp1}

\textbf{File:} \texttt{procedures/fitting.py}

For each \texttt{*\_Steps.json} in \texttt{input\_dir}:

\begin{enumerate}
    \item Flattens all step points into a single array.
    \item Applies \texttt{DataTransformer.apply} if \texttt{fix\_rules} is set.
    \item Fits a 2D polynomial of degree \texttt{high\_degree} and a degree-1 tilt baseline.
    \item Subtracts baseline from the high-degree model via \texttt{ModelOps.subtract\_coeffs}.
    \item Writes \texttt{\{side\_id\}.json} with the flattened model and points.
\end{enumerate}

The degree-1 subtraction removes measurement tilt, isolating the residual surface shape. The constant \texttt{\_LOW\_DEGREE = 1} is defined at module level and shared across Exp1 and Exp2 fitting.

\subsection{fit\_exp2}
\label{subsec:fit_exp2}

\textbf{File:} \texttt{procedures/fitting.py}

For each \texttt{*\_Raw.json} in \texttt{input\_dir}:

\begin{enumerate}
    \item Applies \texttt{DataTransformer.apply} if \texttt{fix\_rules} is set.
    \item Fits a 1D polynomial of degree \texttt{high\_degree} with optional normalization and Ridge regularization.
    \item Fits a degree-1 baseline and subtracts it.
    \item Writes \texttt{\{sample\_id\}.json} with the flattened model and points.
\end{enumerate}

\subsection{compare\_exp1}
\label{subsec:compare_exp1}

\textbf{File:} \texttt{procedures/comparison.py}

Reads \texttt{Side1.json} and \texttt{Side2.json} from \texttt{input\_dir}, averages their polynomial coefficients via \texttt{ModelOps.average\_models}, and writes \texttt{Average.json}. This is the final step of the Exp1 pipeline, producing a single reference surface from both measurement sides.

\subsection{run\_validation}
\label{subsec:validation}

\textbf{File:} \texttt{procedures/validation.py}

Launches the \texttt{PointCloudViewer} GUI in a blocking Tkinter mainloop. Execution resumes only after the user closes the window. Called between preprocessing and fitting in both pipelines to allow manual removal of remaining outliers before fitting.

\begin{lstlisting}[language=Python]
from exp_process.procedures.validation import run_validation
run_validation(output_dir="data/processed/exp1")
\end{lstlisting}
\section{Pipeline}
\label{sec:pipeline}

\textbf{Module path:} \texttt{src/exp\_process/pipeline/}

The pipeline layer is the single entry point for running the full processing sequence. It composes the procedure functions from Section~\ref{sec:procedures} into an ordered, experiment-specific workflow, and exposes one method — \texttt{run()} — as the public interface.

The separation of responsibilities is strict:
\begin{itemize}
    \item \textbf{Core} (Section~\ref{sec:core}) — stateless algorithms, no file I/O.
    \item \textbf{Procedures} (Section~\ref{sec:procedures}) — named steps, each reads and writes files.
    \item \textbf{Pipeline} — assembles steps into a sequence and manages data directories between them.
\end{itemize}

A pipeline object never calls core classes directly; it only calls procedures. This means if the internal algorithm of a step changes, only the relevant core module and possibly the procedure need to change — the pipeline stays untouched.

% ---------------------------------------------------------------------------
\subsection{BasePipeline}
\label{subsec:base_pipeline}
% ---------------------------------------------------------------------------

\textbf{File:} \texttt{pipeline/base.py}

\texttt{BasePipeline} is an abstract base class (\texttt{ABC}) that defines the execution contract for all experiment pipelines. It implements one concrete method — \texttt{run()} — and declares three abstract methods that subclasses must implement.

\textbf{Execution sequence in \texttt{run()}:}
\begin{enumerate}
    \item \textbf{STEP 1 — Preprocess:} calls \texttt{self.\_preprocess()}, which returns the output directory path for the next step.
    \item \textbf{STEP 2 — Validation (GUI):} passes that directory to \texttt{run\_validation()}, blocking until the user closes the viewer.
    \item \textbf{STEP 3 — Fit:} calls \texttt{self.\_fit()}, which reads from the (now manually validated) preprocessing output.
    \item \textbf{STEP 4 — Compare:} calls \texttt{self.\_compare()}, which defaults to a no-op; only \texttt{SurfacePipeline} overrides it.
\end{enumerate}

Each step prints a labelled header (\texttt{=== STEP N ===}) to stdout so progress is visible during a run.

\textbf{Abstract interface:}
\begin{table}[ht]
\centering
\begin{tabular}{lll}
\hline
\textbf{Method} & \textbf{Returns} & \textbf{Description} \\
\hline
\texttt{\_preprocess()} & \texttt{str} & Run preprocessing; return output directory path. \\
\texttt{\_fit()}        & \texttt{None} & Run fitting on preprocessing output. \\
\texttt{\_compare()}    & \texttt{None} & Optional post-fit comparison step. Base no-op. \\
\hline
\end{tabular}
\caption{Abstract methods of \texttt{BasePipeline}.}
\end{table}

\textbf{Where to change:}
\begin{itemize}
    \item \textbf{Skip validation in automated runs:} the GUI call is hard-coded in \texttt{run()}. To skip it, override \texttt{run()} in the subclass and call the steps directly without \texttt{run\_validation}. This is the recommended approach for batch or CI runs.
    \item \textbf{Add a new stage between fit and compare:} add a \texttt{\_post\_fit()} abstract method to \texttt{BasePipeline}, call it inside \texttt{run()} after \texttt{\_fit()}, and provide a default no-op implementation. Subclasses can then override it as needed.
\end{itemize}

% ---------------------------------------------------------------------------
\subsection{SurfacePipeline — Experiment 1}
\label{subsec:surface_pipeline}
% ---------------------------------------------------------------------------

\textbf{File:} \texttt{pipeline/surface.py}

\texttt{SurfacePipeline} implements the full 4-step sequence for Exp1 (T-Shape, 3D surface). It uses \texttt{Exp1PipelineConfig} as its single constructor argument.

\textbf{Config structure:}
\begin{lstlisting}[language=Python]
from exp_process.pipeline.surface import SurfacePipeline, Exp1PipelineConfig
from exp_process.procedures.preprocess import Exp1PreprocessConfig
from exp_process.procedures.fitting import Exp1FittingConfig
from exp_process.procedures.preprocess import BasePreprocessConfig

cfg = Exp1PipelineConfig(
    preprocess=Exp1PreprocessConfig(
        input_dir="data/raw/exp1",
        output_dir="data/processed/exp1/preprocess",
    ),
    fitting=Exp1FittingConfig(
        input_dir="data/processed/exp1/preprocess",
        output_dir="data/processed/exp1/fitting",
        high_degree=4,
    ),
    comparison=BasePreprocessConfig(
        input_dir="data/processed/exp1/fitting",
        output_dir="data/processed/exp1/comparison",
    ),
)

SurfacePipeline(cfg).run()
\end{lstlisting}

\begin{table}[ht]
\centering
\begin{tabular}{lll}
\hline
\textbf{Config field} & \textbf{Type} & \textbf{Purpose} \\
\hline
\texttt{preprocess} & \texttt{Exp1PreprocessConfig} & Parser, cleaning and segmentation settings. \\
\texttt{fitting}    & \texttt{Exp1FittingConfig}    & Polynomial degree and transform rules. \\
\texttt{comparison} & \texttt{BasePreprocessConfig} & Directories for Side1/Side2 averaging. \\
\hline
\end{tabular}
\caption{\texttt{Exp1PipelineConfig} fields.}
\end{table}

Note that \texttt{fitting.input\_dir} must point to the \emph{same} directory as \texttt{preprocess.output\_dir}, and \texttt{comparison.input\_dir} must match \texttt{fitting.output\_dir}. These connections are not enforced automatically — the paths must be consistent in the config.

\textbf{Where to change:}
\begin{itemize}
    \item \textbf{Paths are mismatched between steps:} ensure \texttt{preprocess.output\_dir}, \texttt{fitting.input\_dir}, \texttt{fitting.output\_dir}, and \texttt{comparison.input\_dir} form an unbroken chain.
    \item \textbf{Skip the comparison step:} override \texttt{\_compare()} in a subclass and make it a no-op, or simply do not use the comparison output in downstream scripts.
\end{itemize}

% ---------------------------------------------------------------------------
\subsection{CurvePipeline — Experiment 2}
\label{subsec:curve_pipeline}
% ---------------------------------------------------------------------------

\textbf{File:} \texttt{pipeline/curve.py}

\texttt{CurvePipeline} implements the 3-step sequence for Exp2 (rectangular, 1D curve). The comparison step is not present — Exp2 has no two-side averaging. The \texttt{\_compare()} method inherits the no-op from \texttt{BasePipeline}.

\textbf{Config structure:}
\begin{lstlisting}[language=Python]
from exp_process.pipeline.curve import CurvePipeline, Exp2PipelineConfig
from exp_process.procedures.preprocess import Exp2PreprocessConfig
from exp_process.procedures.fitting import Exp2FittingConfig

cfg = Exp2PipelineConfig(
    preprocess=Exp2PreprocessConfig(
        input_dir="data/raw/exp2",
        output_dir="data/processed/exp2/preprocess",
    ),
    fitting=Exp2FittingConfig(
        input_dir="data/processed/exp2/preprocess",
        output_dir="data/processed/exp2/fitting",
        high_degree=2,
        ridge_alpha=1.0,
    ),
)

CurvePipeline(cfg).run()
\end{lstlisting}

\begin{table}[ht]
\centering
\begin{tabular}{lll}
\hline
\textbf{Config field} & \textbf{Type} & \textbf{Purpose} \\
\hline
\texttt{preprocess} & \texttt{Exp2PreprocessConfig} & Parser and column index settings. \\
\texttt{fitting}    & \texttt{Exp2FittingConfig}    & Polynomial degree, normalization, Ridge $\lambda$. \\
\hline
\end{tabular}
\caption{\texttt{Exp2PipelineConfig} fields.}
\end{table}

% ---------------------------------------------------------------------------
\subsection{Adding a New Pipeline}
\label{subsec:pipeline_extend}
% ---------------------------------------------------------------------------

To support a new experiment type:

\begin{enumerate}
    \item Create a new file \texttt{pipeline/my\_experiment.py}.
    \item Define a \texttt{@dataclass} config aggregating the necessary procedure configs.
    \item Subclass \texttt{BasePipeline}, implement \texttt{\_preprocess()} and \texttt{\_fit()}, and optionally \texttt{\_compare()}.
    \item Return the preprocessing output directory as a \texttt{str} from \texttt{\_preprocess()} so the validation step receives the correct path.
    \item Expose the new class in \texttt{pipeline/\_\_init\_\_.py}.
\end{enumerate}

The minimum skeleton:
\begin{lstlisting}[language=Python]
# pipeline/my_experiment.py
from .base import BasePipeline
from ..procedures.preprocess import MyPreprocessConfig, preprocess_my
from ..procedures.fitting import MyFittingConfig, fit_my

@dataclass
class MyPipelineConfig:
    preprocess: MyPreprocessConfig
    fitting: MyFittingConfig

class MyPipeline(BasePipeline):
    def __init__(self, cfg: MyPipelineConfig):
        self.cfg = cfg

    def _preprocess(self) -> str:
        preprocess_my(self.cfg.preprocess)
        return str(self.cfg.preprocess.output_dir)

    def _fit(self) -> None:
        fit_my(self.cfg.fitting)
\end{lstlisting}

\section{Utilities and GUI}
\label{sec:utils_gui}

This section covers the two support modules of \texttt{exp\_process}: the I/O utilities in \texttt{utils/io.py} and the interactive point cloud viewer in \texttt{gui/viewer.py}. Neither module contains domain logic — they provide the plumbing that all other layers rely on.

% ---------------------------------------------------------------------------
\subsection{IOUtils — \texttt{utils/io.py}}
\label{subsec:io_utils}
% ---------------------------------------------------------------------------

\textbf{File:} \texttt{utils/io.py}

All JSON read and write operations in the module go through \texttt{IOUtils}. The class exists to centralise two concerns: handling NumPy types during serialisation, and providing a consistent file-naming convention for pipeline output.

\subsubsection{NumpyEncoder}

A custom \texttt{json.JSONEncoder} subclass that transparently converts NumPy scalars and arrays to native Python types before serialisation:

\begin{table}[ht]
\centering
\begin{tabular}{ll}
\hline
\textbf{NumPy type} & \textbf{Serialised as} \\
\hline
\texttt{np.integer}  & \texttt{int} \\
\texttt{np.floating} & \texttt{float} \\
\texttt{np.ndarray}  & \texttt{list} (via \texttt{.tolist()}) \\
\hline
\end{tabular}
\caption{\texttt{NumpyEncoder} type mapping.}
\end{table}

This encoder is used automatically by all \texttt{save\_json} calls. It does not need to be imported or called directly — it is an implementation detail of the I/O layer.

\subsubsection{Module-level vs.\ class-level functions}

\texttt{io.py} exposes both module-level functions (\texttt{save\_json}, \texttt{load\_json}) and the same methods as static methods on \texttt{IOUtils}. The only behavioural difference is in \texttt{load\_json}:

\begin{itemize}
    \item Module-level \texttt{load\_json(filepath)} — raises \texttt{FileNotFoundError} if the file does not exist.
    \item \texttt{IOUtils.load\_json(filepath)} — returns \texttt{None} if the file does not exist (safe for pipeline use).
\end{itemize}

All procedure modules use \texttt{IOUtils} (the class), so missing files produce a \texttt{[SKIP]} rather than a crash. The module-level functions are kept for standalone scripts or external callers that prefer an exception.

\subsubsection{IOUtils.save\_result}

The primary write method used by all procedures:

\begin{lstlisting}[language=Python]
IOUtils.save_result(output_dir: Path, name: str, data: dict) -> Path
\end{lstlisting}

Constructs the output path as \texttt{output\_dir / name.json}, serialises \texttt{data} with \texttt{NumpyEncoder}, creates \texttt{output\_dir} if it does not exist, and returns the resolved \texttt{Path}. The \texttt{name} argument is always the item identifier without extension: \texttt{"Side1"}, \texttt{"Side1\_Steps"}, \texttt{"Average"}, etc.

\textbf{Where to change:}
\begin{itemize}
    \item \textbf{Different output format (e.g.\ CSV, HDF5):} replace \texttt{IOUtils.save\_result} calls in the relevant procedure. The core and pipeline layers do not call \texttt{save\_result} directly, so the change is isolated to the procedures.
    \item \textbf{Custom file naming convention:} the \texttt{name} argument is controlled by each procedure. Change it there — \texttt{IOUtils} does not impose a naming scheme beyond the \texttt{.json} extension.
    \item \textbf{Adding logging:} \texttt{save\_json} already calls \texttt{logger.info} on success. To change the log level or destination, configure the \texttt{logging} module at the application entry point; no changes to \texttt{io.py} are needed.
\end{itemize}

% ---------------------------------------------------------------------------
\subsection{PointCloudViewer — \texttt{gui/viewer.py}}
\label{subsec:point_cloud_viewer}
% ---------------------------------------------------------------------------

\textbf{File:} \texttt{gui/viewer.py}

\texttt{PointCloudViewer} is a Tkinter + Matplotlib GUI for manually inspecting and editing point cloud JSON files between the preprocessing and fitting steps. It is launched exclusively through \texttt{run\_validation()} (Section~\ref{subsec:validation}).

\subsubsection{Layout}

The window is split into two panels:
\begin{itemize}
    \item \textbf{Left panel} — Matplotlib canvas with the full navigation toolbar (zoom, pan, save figure). Displays the currently selected step or flat point cloud, with a polynomial model overlay when coefficients are present in the file.
    \item \textbf{Right panel} — file/sample selector (dropdown), step list (listbox), instructions label, \textit{Save Changes} button, and an info label showing point count.
\end{itemize}

\subsubsection{Data types}

When loading files from \texttt{input\_dir}, the viewer classifies each JSON file into one of three types based on its structure:

\begin{table}[ht]
\centering
\begin{tabular}{lll}
\hline
\textbf{Type} & \textbf{Detected when} & \textbf{Example files} \\
\hline
\texttt{steps} & top-level \texttt{"steps"} key is a list & \texttt{Side1\_Steps.json} \\
\texttt{flat}  & top-level \texttt{"points"} key is a list & \texttt{01\_Raw.json} \\
\texttt{model} & neither key present & \texttt{Side1.json}, \texttt{Average.json} \\
\hline
\end{tabular}
\caption{File type classification in \texttt{PointCloudViewer}.}
\end{table}

\texttt{model} files are loaded and shown in the dropdown but are not editable — they contain coefficient data, not raw points. The viewer overlays the fitted polynomial curve on the plot when it reads a \texttt{"coeffs"} key in the currently displayed file.

On startup, the viewer auto-selects the first \texttt{steps} file found; if none exist, it selects the first file alphabetically.

\subsubsection{Interactive editing}

The core editing interaction is click-to-delete:
\begin{enumerate}
    \item Select a file from the dropdown.
    \item Select a step from the listbox (or \textit{All Points} for flat files).
    \item Click on a point in the canvas to remove it. The point nearest to the click within a picker tolerance of 5 pixels is deleted.
    \item After editing, click \textit{Save Changes} to overwrite the JSON file on disk. The \texttt{modified} flag is set on any deletion and cleared on save.
\end{enumerate}

Deletions modify \texttt{data\_store} in memory. Nothing is written to disk until \textit{Save Changes} is clicked. Closing the window without saving discards all deletions.

\subsubsection{Model overlay}

When the selected file contains a \texttt{"coeffs"} key (i.e.\ a fitted model), the viewer renders the polynomial curve over the point cloud. The overlay is reconstructed each time \texttt{update\_plot()} is called from the stored axes limits, so it remains valid after point deletions change the scale.

This also means that if you open a \texttt{model} file (e.g.\ \texttt{Side1.json} after fitting), you can visually verify the fit quality without running a separate script.

\textbf{Where to change:}
\begin{itemize}
    \item \textbf{Picker tolerance too tight or too loose:} the value \texttt{picker=5} is hard-coded in \texttt{\_setup\_figure()}. Increase it if clicks are not registering; decrease it if adjacent points are accidentally deleted.
    \item \textbf{Plot axes labels:} currently fixed to \texttt{"Y (mm)"} and \texttt{"Z (mm)"} in \texttt{\_setup\_figure()} and \texttt{update\_plot()}. If the data coordinate system changes, update both locations.
    \item \textbf{Add undo support:} the viewer has no undo. To add it, keep a stack of point arrays in \texttt{update\_current\_points()} before overwriting, and add an \textit{Undo} button that pops the stack and redraws.
    \item \textbf{View 3D surface data (Exp1):} the viewer currently projects onto a 2D Y/Z plane. To visualize the full 3D point cloud, replace \texttt{fig.add\_subplot(111)} with \texttt{fig.add\_subplot(111, projection='3d')} and adjust the scatter call — note this requires updating both \texttt{\_setup\_figure()} and \texttt{update\_plot()}.
\end{itemize}
