\chapter{Experimental Data Management}
\label{ch:exp_data}

\textbf{Module Path:} \texttt{src/Exp\_Data/}

This module serves as the central repository and configuration manager for the project. Its primary purpose is to bridge the gap between the physical experimental samples and their digital twins. By managing raw measurement data and simulation parameters, it ensures that the Finite Element models are generated based on the \textit{actual} mean dimensions of the manufactured specimens, rather than theoretical CAD values.

\section{Module Architecture}

The module is structured to support multiple experimental campaigns, isolating the data logic for each specific sample geometry.

\subsection{Sub-Packages}
\begin{itemize}
    \item \textbf{Experiment 1 (\texttt{s1\_exp}):} 
    Focuses on the standard T-Shape geometry used for the primary Residual Stress benchmark. It handles the averaging of flange and web measurements to define the parameterized model.
    
    \item \textbf{Experiment 2 (\texttt{s2\_exp}):} 
    Dedicated to the Profile/Milling experiment. It manages the dimensions for the milling plate and the specific cut geometries required for the sensitivity analysis (to be detailed in subsequent sections).
\end{itemize}

\subsection{Core Functionalities}
Across both experiments, the module provides three key services to the framework:

\begin{enumerate}
    \item \textbf{Dimensional Averaging:} 
    Scripts like \texttt{sample\_dim.py} and \texttt{mean\_dim.py} ingest lists of raw caliper measurements (e.g., widths, thicknesses at multiple points) and calculate the statistical mean. These values are then exported as a dictionary to be consumed by the Geometry Generators.
    
    \item \textbf{Configuration Management:} 
    The \texttt{write\_input.py} script defines the schema for the \texttt{config.json} file. It validates parameters such as Young's Modulus, mesh density ranges, and directory paths, ensuring the simulation pipeline receives valid inputs.
    
    \item \textbf{User Interface (GUI):} 
    To facilitate parameter tuning without editing JSON files manually, the module includes a graphical interface (\texttt{write\_input\_gui.py}). This tool allows users to modify material properties, define DOE ranges (Design of Experiments), and save valid configurations visually.
\end{enumerate}

\section{Experiment 1: T-Shape Benchmark}
\label{sec:exp_s1}

\textbf{Package:} \texttt{src/Exp\_Data/s1\_exp/}

This sub-package manages the data for the primary Residual Stress benchmark (T-Shape). It includes not only the dimensional data but also the configuration infrastructure for the entire simulation framework.

\subsection{Dimensional Averaging}
\textbf{Scripts:} \texttt{sample\_dim.py}, \texttt{mean\_dim.py}

To ensure the finite element model represents the physical reality, raw caliper measurements from multiple samples are stored and averaged.
\begin{itemize}
    \item \textbf{Raw Data (\texttt{sample\_dim.py}):} Contains lists of measurements for critical features: \texttt{bottom\_width}, \texttt{wall\_width}, \texttt{total\_height}, etc.
    \item \textbf{Mean Calculation (\texttt{mean\_dim.py}):} 
    The function \texttt{DimOne()} aggregates these lists to compute the statistical mean.
    \begin{equation}
        \mu_{width} = \frac{1}{N} \sum_{i=1}^{N} w_i
    \end{equation}
    It creates a dictionary \texttt{Mean\_dim\_wp} which is imported by the Abaqus geometry scripts (Module II) to construct the parametric T-shape.
\end{itemize}

\section{Configuration Interface (User Frontend)}
\label{sec:config_interface}

\textbf{Scripts:} \texttt{write\_input.py}, \texttt{write\_input\_gui.py}

Since manual editing of JSON files is prone to syntax errors, this module provides a robust frontend for simulation management. It consists of a logic layer for validation and a graphical layer for user interaction.

\subsection{Backend Logic and Validation}
\textbf{Class:} \texttt{AbaqusParameters} (in \texttt{write\_input.py})

This class acts as the gatekeeper for simulation integrity. It manages the lifecycle of the configuration data, ensuring that invalid parameters are caught before the simulation pipeline is triggered.

\begin{itemize}
    \item \textbf{Default Initialization:} 
    If a configuration file is missing, the class automatically generates a complete structure with safe default values (e.g., standard steel properties, conservative time steps), ensuring the system is always in a runnable state.
    
    \item \textbf{Physical Constraints Enforcement:} 
    The \texttt{validate\_parameters()} method implements strict physical checks:
    \begin{itemize}
        \item \textbf{Time Stepping:} Ensures \texttt{initialInc}, \texttt{maxInc}, and \texttt{timePeriod} are strictly positive to prevent solver crashes.
        \item \textbf{Directory Existence:} Verifies if the \texttt{work\_directory} actually exists on the disk, alerting the user immediately if a path is invalid.
        \item \textbf{Post-Processing Metadata:} Checks for required keys in the \texttt{conversion\_params} block (e.g., \texttt{MeshType}, \texttt{Piecenum}), which are critical for the correct parsing of ODB files later in the pipeline.
    \end{itemize}
\end{itemize}

\subsection{Graphical User Interface (GUI)}
\textbf{Class:} \texttt{AbaqusConfigGUI} (in \texttt{write\_input\_gui.py})

Built with \texttt{customtkinter}, this application provides a modern, dark-mode friendly dashboard for controlling the experiment. It abstracts the underlying JSON structure into organized input fields.

\begin{itemize}
    \item \textbf{Real-Time Synchronization:} 
    The interface automatically loads existing configurations on startup and maps them to the visual widgets. Any change in the UI is held in memory until explicitly saved.
    
    \item \textbf{Categorized Editing:} 
    Parameters are grouped logically to aid the user workflow:
    \begin{itemize}
        \item \textbf{Material:} Fine-tune Elastic Modulus ($E$), Poisson's ratio ($\nu$), and Density ($\rho$).
        \item \textbf{DOE Settings:} Define the search space for the parametric study by setting Min/Max ranges for Mesh Size and Specimen Length.
        \item \textbf{Solver Control:} Adjust convergence parameters (increments) without needing to know Abaqus keyword syntax.
    \end{itemize}
    
    \item \textbf{Safety Features:} 
    Includes a "Validate" button that runs the backend checks and displays specific error messages (e.g., "Max increment must be positive") via pop-ups, preventing the saving of broken configurations.
\end{itemize}

\subsection{Global Configuration Storage}
\textbf{File:} \texttt{config.json}

This JSON file serves as the "Single Source of Truth" for the entire framework. All modules—from the Geometry Generator (Part II) to the Result Converter (Part I)—read from this file, guaranteeing consistency.

\begin{itemize}
    \item \textbf{Directories:} Defines absolute paths for \texttt{work\_directory}, \texttt{CM\_directory}, and \texttt{REA\_directory}, ensuring all scripts operate in the correct workspace.
    \item \textbf{Mesh Settings:} Defines the \texttt{mesh\_step} and \texttt{length\_step}, which control the granularity of the sensitivity analysis (e.g., testing mesh sizes from 0.78mm to 0.79mm with 0.05mm steps).
    \item \textbf{Conversion Params:} Stores specific metadata (e.g., \texttt{BeginFrame}, \texttt{EndFrame}) used by the generic converters to slice the simulation time history correctly during data extraction.
\end{itemize}
\section{Experiment 2: Milling Profile Analysis}
\label{sec:exp_s2}

\textbf{Package:} \texttt{src/Exp\_Data/s2\_exp/}

This sub-package handles the dimensional data for the Milling experiment (Module B). Unlike the standard T-shape, this experiment often involves prismatic plates or specific cut geometries for sensitivity analysis.

\subsection{Data Structure}
\textbf{Scripts:} \texttt{sample\_dim\_2.py}, \texttt{mean\_dim\_2.py}

Following the architecture of Experiment 1, this module isolates the data specific to the second experimental campaign.

\begin{itemize}
    \item \textbf{Raw Data (\texttt{sample\_dim\_2.py}):} 
    Stores the measurement lists specific to the milling specimens (e.g., plate thickness, cut depth variations).
    
    \item \textbf{Mean Calculation (\texttt{mean\_dim\_2.py}):} 
    The function \texttt{DimTwo()} (analogous to \texttt{DimOne}) computes the average dimensions. These values are consumed by the \texttt{sim\_two} and \texttt{sim\_iv} geometry strategies in the Abaqus modules to build the corresponding FE models.
\end{itemize}