\section{Stage 1: Configuration and Orchestration}
\label{sec:conv_orchestration}

\textbf{Scripts:} \texttt{Odb\_Npz\_Parameters.py}, \texttt{ODB\_2\_XDMF.py}

The conversion process is initiated by the orchestration layer, which manages configuration loading, directory discovery, and batch processing logic.

\subsection{Parameter Management}
The class \texttt{ODB2NPYParameters} acts as the bridge between the project's global \texttt{config.json} and the conversion scripts.
\begin{itemize}
    \item \textbf{Config Loading:} It reads the global configuration to determine the root directories for specific simulation methods (e.g., "Contour Method").
    \item \textbf{Output Paths:} It automatically defines the output structure, creating a \texttt{npy\_files} directory within the simulation folder to store intermediate results.
\end{itemize}

\subsection{Batch Processing}
The \texttt{OdbBatchConverter} class handles the iteration over multiple simulation files:
\begin{itemize}
    \item \textbf{Discovery:} It scans the target directory for all files ending in \texttt{.odb}.
    \item \textbf{Structure Inspection:} Before conversion, it can inspect the ODB to list available Steps and Instances (nodes/elements counts), helping to validate the file integrity.
    \item \textbf{Unicode Handling:} It includes a \texttt{safe\_str\_convert} utility to handle potential encoding issues when passing file paths between modern Python environments and the older Abaqus Python kernel.
\end{itemize}