\section{Stage 3: Field Interpolation}
\label{sec:ep_interpolation}

\textbf{Script:} \texttt{interpolator.py}

Maps a source stress field (analytical or Abaqus-format) onto the target finite element mesh by interpolating at each element centroid.

\subsection{Input Loading}

\texttt{ElementTensionInterpolator} reads both the target mesh and the source stress field from \texttt{output\_dir}. File detection is automatic: each load method first looks for a standard filename (\texttt{elements\_data.txt}, \texttt{residual\_stress.txt}) and, if not found, scans the directory for a suitable candidate.

\texttt{load\_tension\_field} supports two input formats detected automatically from the file header:

\begin{table}[ht]
\centering
\begin{tabular}{lll}
\hline
\textbf{Format} & \textbf{Detected by} & \textbf{Columns} \\
\hline
Abaqus & header contains \texttt{INITIAL CONDITIONS} & ID, S11, S22, S33, S12, S13, S23 \\
Cylindrical & header contains \texttt{Sigma\_r} & ID, X, Y, Z, R, $\theta$, $\sigma_r$, $\sigma_\theta$, $\sigma_z$, $\tau_{rt}$, $\tau_{rz}$, $\tau_{\theta z}$ \\
\hline
\end{tabular}
\caption{Source stress file formats accepted by \texttt{load\_tension\_field}.}
\end{table}

For cylindrical format, the components are converted to Cartesian in-place before interpolation:
\begin{align*}
    S_{11} &= \sigma_r \cos^2\theta + \sigma_\theta \sin^2\theta \\
    S_{12} &= (\sigma_r - \sigma_\theta)\sin\theta\cos\theta
\end{align*}

\subsection{Interpolation Strategy}

For each of the six stress components independently:
\begin{enumerate}
    \item \textbf{Linear (primary):} \texttt{LinearNDInterpolator} over the source point cloud in 3D.
    \item \textbf{Nearest-neighbor (fallback for NaN):} \texttt{NearestNDInterpolator} fills elements outside the convex hull of the source data.
    \item \textbf{Mean (last resort):} if both interpolators fail, the component is filled with the mean value across all source points.
\end{enumerate}

If fewer than 4 source points are available (minimum for 3D linear interpolation), all elements receive the mean stress vector directly.

\subsection{Output}

\texttt{generate\_interpolated\_tension\_file} writes \texttt{interpolated\_element\_stresses.txt} formatted as an Abaqus initial condition:
\begin{lstlisting}
*INITIAL CONDITIONS, TYPE=STRESS
element_id, S11, S22, S33, S12, S13, S23
...
\end{lstlisting}