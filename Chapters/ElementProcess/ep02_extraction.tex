\section{Stage 1: Element Extraction}
\label{sec:ep_extraction}

\textbf{Script:} \texttt{s1\_Ele\_Extractor.py}

The first stage of the pipeline converts the unstructured text data from Abaqus input files into structured tabular data (CSV/Pandas DataFrames) suitable for mathematical processing.

\subsection{Input Parsing Logic}
The script manually parses the \texttt{.inp} text format to identify geometric definitions:

\begin{itemize}
    \item \textbf{Element Parsing:} 
    The function \texttt{extract\_elements\_from\_inp} scans for the \texttt{*ELEMENT} keyword. It extracts the Element ID, the Element Type (e.g., C3D8R), and the list of connected Node IDs that define the element's topology.
    
    \item \textbf{Coordinate Extraction:} 
    A companion function (referenced as \texttt{extract\_node\_coordinates}) retrieves the spatial $(X, Y, Z)$ coordinates for every node defined under the \texttt{*NODE} keyword.
\end{itemize}

\subsection{Geometry Calculation}
To facilitate field interpolation, the script calculates the **centroid** of each element.
\begin{equation}
    C_{elem} = \frac{1}{N} \sum_{i=1}^{N} P_{node\_i}
\end{equation}
Where $P_{node\_i}$ are the coordinates of the nodes connected to the element. This reduces the element to a single point in space $(X_{center}, Y_{center}, Z_{center})$.

\subsection{Output}
The processed data is saved to the output directory as a tab-separated file containing:
\begin{itemize}
    \item Element ID
    \item Element Type
    \item Centroid Coordinates ($X, Y, Z$)
\end{itemize}
This file serves as the geometric basis for the subsequent stress field generation.