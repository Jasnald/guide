\section{Workflow Orchestration}
\label{sec:ep_orchestration}

\textbf{Script:} \texttt{elements\_main.py}

\textbf{Module path:} \texttt{src/element\_process/}

The module was restructured and all scripts renamed to the lowercase-underscore convention:

\begin{table}[ht]
\centering
\begin{tabular}{ll}
\hline
\textbf{Old name} & \textbf{New name} \\
\hline
\texttt{Elements\_main.py}   & \texttt{elements\_main.py} \\
\texttt{s1\_Ele\_Extractor.py} & \texttt{extractor.py} \\
\texttt{s2\_RE\_Field.py}      & \texttt{field\_analitic.py} \\
\texttt{s3\_RE\_Interpolator.py} & \texttt{interpolator.py} \\
\texttt{s2\_RE\_ExnCon.py}    & \texttt{stress\_mapping.py} \\
\texttt{s2\_RE\_ExnCon2.py}   & \texttt{stress\_mapping\_2.py} \\
--- (new)                    & \texttt{elements\_plot.py} \\
\hline
\end{tabular}
\caption{\texttt{element\_process} module file renaming.}
\end{table}

\subsection{Analytical Pipeline}
Used for validation or when experimental data is unavailable. Runs three stages in sequence:

\begin{enumerate}
    \item \textbf{Extraction:} Parses geometry and topology from the \texttt{.inp} file via \texttt{extractor.py}.
    \item \textbf{Synthetic Generation:} Generates a theoretical cylindrical stress field from the mesh bounding box via \texttt{field\_analitic.py}.
    \item \textbf{Interpolation:} Maps the synthetic point cloud onto element centroids via \texttt{interpolator.py}.
\end{enumerate}

\subsection{Data-Driven Pipeline}
For real simulation data (Contour Method or RSA), the generation step is replaced by direct HDF5 mapping:

\begin{enumerate}
    \item \textbf{Extraction:} Same as above; prepares target mesh centroids.
    \item \textbf{Direct Mapping:} Reads the HDF5 source (\texttt{S\_batch.h5}) and maps stresses to element centroids using KDTree via \texttt{stress\_mapping.py} or \texttt{stress\_mapping\_2.py}.
\end{enumerate}