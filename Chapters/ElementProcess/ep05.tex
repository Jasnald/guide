\section{Data-Driven Stress Mapping}
\label{sec:ep_datadriven}

\textbf{Script:} \texttt{stress\_mapping.py}

Maps $S_{33}$ stress from the \texttt{S\_batch.h5} HDF5 source onto the target finite element mesh using a planar KDTree strategy. Used in the data-driven pipeline (Contour Method results).

\subsection{Constructor Parameters}

\begin{table}[ht]
\centering
\begin{tabular}{lll}
\hline
\textbf{Parameter} & \textbf{Default} & \textbf{Description} \\
\hline
\texttt{base\_dir}   & ---                & Directory containing \texttt{.inp} and \texttt{Output/} folders. \\
\texttt{tolerance}  & \texttt{5e-2}     & $|z|$ threshold for the $z{=}0$ plane extraction. \\
\texttt{chunk\_size} & \texttt{10000}    & Batch size for HDF5 read operations on large datasets. \\
\hline
\end{tabular}
\caption{\texttt{StressProcessor} constructor parameters.}
\end{table}

\subsection{Mapping Strategy}

The workflow uses a 2D KDTree in the $x$-$y$ plane rather than a full 3D tree. This is intentional: the Contour Method extracts a cut-plane measurement at $z = 0$, which is then projected onto all $z$-planes of the mesh.

\begin{enumerate}
    \item \texttt{read\_combined\_hdf5\_from\_folder} --- reads \texttt{S\_batch.h5} and extracts coordinates and \texttt{stress\_tensor[:, 8]} ($S_{33}$, index 8 in the 9-component tensor) for the first available step/frame.
    \item \texttt{extract\_z0\_data\_by\_z} --- filters HDF5 nodes where $|z| <$ \texttt{tolerance} using vectorised \texttt{np.isclose}, returning the $z{=}0$ slice.
    \item \texttt{create\_stress\_mapping\_by\_z} --- for each unique $Z$ plane in the mesh, builds a KDTree on $(x, y)$ of the $z{=}0$ source data and queries the centroid coordinates of all elements in that plane. KDTrees are cached by array hash to avoid rebuilding for repeated queries.
\end{enumerate}

\subsection{Output Formats}

\texttt{save\_organized\_data} and \texttt{create\_abaqus\_stress\_file} produce three output files per simulation:

\begin{table}[ht]
\centering
\begin{tabular}{ll}
\hline
\textbf{File} & \textbf{Format} \\
\hline
\texttt{stress\_mapping\_by\_z.json} & Full mapping with element coordinates and distances. \\
\texttt{stress\_mapping\_by\_z.h5}   & Same structure in HDF5 for efficient downstream reads. \\
\texttt{stress\_input.txt}           & Abaqus-format: \texttt{element\_id, 0, 0, S33, 0, 0, 0} per line. \\
\hline
\end{tabular}
\caption{Output files from \texttt{stress\_mapping.py}.}
\end{table}

\subsection{Usage}

\begin{lstlisting}[language=Python]
from element_process.stress_mapping import StressProcessor

processor = StressProcessor(
    base_dir="C:/Simulations/Residual_Stresses_Analysis",
    tolerance=5e-2,
    chunk_size=10000
)
results = processor.process_all_simulations(
    hdf5_folder="C:/Simulations/Contour_Method/xdmf_hdf5_files"
)
\end{lstlisting}