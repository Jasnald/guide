\chapter{Core INP Manipulation Library}
\label{ch:inp_modules}

\textbf{Module Path:} \texttt{src/simulations/inp\_process/}

This module serves as the foundational library for programmatically interacting with Abaqus Input Files (\texttt{.inp}). Unlike standard Python scripts that utilize the Abaqus Object Model (AOM) inside the GUI, this library acts as a standalone parser and modifier, manipulating the ASCII input files directly. This approach offers greater flexibility, performance, and independence from the Abaqus licensing environment during the pre-processing phase.


\textbf{Note:} This is a core module, so the user will rarely need to modify anything here.

\section{Library Architecture}

The library is structured into four functional components that abstract the complexity of finite element definitions.

\subsection{1. Data Structures and Parsing}
\textbf{Files:} \texttt{dataclasses.py}, \texttt{parser.py}

Defines lightweight data structures (e.g., \texttt{Node}, \texttt{Element}, \texttt{SectionProperties}) to represent FE entities in memory. The \texttt{INPParser} class provides robust static methods to handle the case-insensitive and whitespace-variable nature of Abaqus keywords.

\subsection{2. Geometry Extraction}
\textbf{File:} \texttt{process.py}

Responsible for interpreting the physical model described in the input file.
\begin{itemize}
    \item \textbf{Entity Reading:} The \texttt{ReadEntities} class iterates through the file to construct lists of nodes and elements.
    \item \textbf{Section Mapping:} The \texttt{SectionReader} maps geometric properties (like Shell Thickness) to specific element sets.
    \item \textbf{Region Filtering:} Utilities like \texttt{RegionElementExtractor} allow extracting specific subsets of elements based on bounding boxes ($x_{min}, x_{max}$), facilitating localized analysis.
\end{itemize}

\subsection{3. Model Modification (The "Injector")}
\textbf{File:} \texttt{modifier.py}

The core engine for Residual Stress Analysis, allowing the injection of external data into an existing simulation deck.
\begin{itemize}
    \item \textbf{Stress Generation:} The \texttt{InitialStressGenerator} converts dictionary-based stress data into formatted \texttt{*Initial Conditions, type=STRESS} blocks.
    \item \textbf{Safe Insertion:} The \texttt{INPInserter} locates safe injection points within the file structure (e.g., placing Boundary Conditions inside the correct \texttt{*Step}) to ensure the generated file runs without syntax errors.
\end{itemize}

\subsection{4. Execution Wrappers}
\textbf{File:} \texttt{runners.py}

Abstracts the command-line calls to the solver. The \texttt{AbaqusJobRunner} manages the execution of jobs, handling configuration parameters like CPU cores and GPU acceleration flags automatically.

\section{Data Structures and Configuration}
\label{sec:inp_structures}

\textbf{Script:} \texttt{dataclasses.py}

To decouple the Python logic from the specific formatting of Abaqus input files, this module defines strongly-typed data structures (using Python's \texttt{@dataclass}). These classes act as intermediate representations for the physical entities and execution configurations.

\subsection{Finite Element Entities}
\begin{itemize}
    \item \textbf{Node:} Represents a geometric point in 3D space with an ID (label) and coordinates $(x, y, z)$.
    \item \textbf{Element:} Stores the mesh topology, linking an ID (label) to a list of constituent \texttt{Node} IDs and defining the element type (e.g., C3D8R).
    \item \textbf{SectionProperties:} Abstract container for physical properties associated with element sets. It distinguishes between \texttt{Solid} (homogeneous) and \texttt{Shell} sections, storing critical thickness parameters needed for stress calculation integration points.
\end{itemize}

\subsection{Execution Configuration}
\textbf{Class:} \texttt{AbaqusJobConfig}

This class encapsulates all parameters required to launch an Abaqus solver job via the command line. It includes:
\begin{itemize}
    \item \textbf{Resource Allocation:} Number of CPUs (\texttt{n\_cpus}) and memory percentage (\texttt{memory}).
    \item \textbf{Environment Settings:} Path to the Abaqus executable (\texttt{abaqus\_cmd}) and scratch directory usage (\texttt{use\_scratch}).
    \item \textbf{Job Control:} Timeout limits (default 30 hours) to prevent stalled processes from hanging the pipeline indefinitely.
    \item \textbf{Output Mode:} \texttt{silent\_mode} — if \texttt{True}, captures stdout/stderr to a log file; if \texttt{False}, streams output live. \texttt{auto\_cleanup} — if \texttt{True}, automatically removes the scratch directory after a successful run.
\end{itemize}

\textbf{Class:} \texttt{AbaqusScriptConfig}

A companion dataclass for running standalone Abaqus Python scripts (not full solver jobs). Used by \texttt{AbaqusScriptRunner}:
\begin{itemize}
    \item \texttt{script\_name} — path to the \texttt{.py} script to execute.
    \item \texttt{working\_dir} — working directory; UNC paths (\texttt{\textbackslash\textbackslash server\textbackslash share}) are handled transparently via \texttt{pushd}/\texttt{popd}.
    \item \texttt{python\_cmd} — Python interpreter or Abaqus Python wrapper command.
    \item \texttt{env} — optional dictionary of environment variables passed to the subprocess.
\end{itemize}
\section{Robust INP Parsing}
\label{sec:inp_parsing}

\textbf{Script:} \texttt{parser.py}

Abaqus Input Files (\texttt{.inp}) are ASCII-based but often contain inconsistent formatting (variations in capitalization, whitespace, and parameter ordering). The \texttt{INPParser} class provides static utility methods to handle this variability robustly.

\subsection{Key Methods}
\begin{itemize}
    \item \textbf{\texttt{is\_header(line, keyword)}:} 
    Performs a case-insensitive check to see if a line starts with a specific keyword (e.g., \texttt{*ELEMENT}). This centralizes the logic for identifying Abaqus command blocks.
    
    \item \textbf{\texttt{get\_parameter(line, key)}:} 
    Extracts values from comma-separated key-value pairs typical of Abaqus headers.
    \begin{lstlisting}
    Input: "*ELEMENT, TYPE=C3D8R, ELSET=Set-1"
    Call: get_parameter(line, "ELSET")
    Output: "Set-1"
    \end{lstlisting}
    It handles edge cases like spaces around the equals sign or mixed casing, ensuring reliable metadata extraction.
\end{itemize}
\section{Input/Output Operations}
\label{sec:inp_io}

\textbf{Scripts:} \texttt{reader.py}, \texttt{writer.py}

These modules abstract the file system interactions, providing specialized readers for the various file formats encountered in the workflow.

\subsection{Readers}
The \texttt{reader.py} module implements specific classes for data ingestion:
\begin{itemize}
    \item \textbf{\texttt{INPReader}:} A simple wrapper around file reading that ensures consistent encoding (UTF-8) and provides line-by-line access to the input deck.
    \item \textbf{\texttt{JSONReader}:} Used to load the polynomial parameters (degree and coefficients) generated in the Preprocessing phase (Module A/B). It handles the conversion of lists back into NumPy arrays for mathematical operations.
    \item \textbf{\texttt{StressReader}:} A CSV parser designed to read external stress field definitions. It robustly handles comment lines (starting with \texttt{\#}) and converts tabular stress data into a dictionary mapping Element IDs to stress vectors.
\end{itemize}

\subsection{Writers}
The \texttt{writer.py} module contains the \texttt{INPWriter} class. While currently a lightweight wrapper for writing lists of strings to disk, it centralizes file encoding handling, ensuring that the modified Input Files generated by the software are always compliant with the text format expected by the Abaqus solver.
\section{Model Modification and Injection}
\label{sec:inp_modification}

\textbf{Script:} \texttt{modifier.py}

This module acts as the "injector" engine of the library. It is responsible for generating valid Abaqus command blocks (like Element Sets or Initial Conditions) and surgically inserting them into an existing input deck without breaking the file structure.

\subsection{Content Generators}
These classes transform raw data into Abaqus-formatted strings:

\begin{itemize}
    \item \textbf{\texttt{InitialStressGenerator}:} 
    Takes a dictionary of stress tensors and formats them into an \texttt{*Initial Conditions, type=STRESS} block. It iterates through each element and its integration points, ensuring the correct CSV format required by the solver.
    
    \item \textbf{\texttt{ElsetGenerator}:} 
    Automatically creates \texttt{*Elset} definitions for the elements receiving stress. This is crucial because Abaqus applies initial conditions to specific element sets, not global IDs.

    \item \textbf{\texttt{BCGenerator}:} 
    Generates \texttt{*Boundary} cards for displacement control. It calculates the necessary nodal displacements (e.g., to deform a mesh to match a specific shape) and formats them as \texttt{Type: Displacement/Rotation} conditions.
\end{itemize}

\subsection{Intelligent Insertion Logic}
\textbf{Class:} \texttt{INPInserter}

Modifying an INP file requires placing commands in specific sections (e.g., Assembly vs. Step level). The inserter implements context-aware logic:

\begin{itemize}
    \item \textbf{\texttt{insert\_initial\_stresses}:} 
    Scans the file for the first \texttt{*Step} keyword. It injects the stress block \textit{before} the step begins, effectively setting the initial state of the simulation. It also handles merging with existing \texttt{*Predefined Fields} if present.
    
    \item \textbf{\texttt{insert\_elsets}:} 
    Locates the \texttt{*End Assembly} marker to inject element sets inside the assembly definition but outside of instance definitions, ensuring global visibility.
    
    \item \textbf{\texttt{replace\_material\_block}:} 
    Finds a material definition by name and replaces its entire sub-block while preserving the rest of the file. End-of-block detection is done by scanning for a predefined list of "block breaker" keywords (\texttt{*Step}, \texttt{*Boundary}, \texttt{*Node}, \texttt{*Element}, etc.).

    \item \textbf{\texttt{insert\_in\_step}:} 
    Inserts arbitrary lines (e.g., boundary conditions) immediately before \texttt{*End Step} of a \textit{named} step. Useful for adding per-step output requests or displacement controls after the rest of the file has been assembled.

    \item \textbf{\texttt{fix\_restart\_frequency}:} 
    Scans every line and replaces \texttt{frequency=0} with \texttt{frequency=1} in \texttt{*Restart} and \texttt{*Output} blocks. Abaqus ignores output requests with zero frequency, and this often introduces silent errors when input files are exported from the GUI.
\end{itemize}

\subsection{High-Level Writer}
\textbf{Class:} \texttt{StressINPWriter}

This class encapsulates the entire read-modify-write cycle. It orchestrates the readers, generators, and inserters to produce a final, runnable Abaqus input file with the applied residual stress fields.
\section{Geometry Interpretation and Processing}
\label{sec:inp_geometry}

\textbf{Script:} \texttt{process.py}

While the parsing module handles the text syntax, this module is responsible for reconstructing the semantic meaning of the finite element model. It links abstract element definitions to physical properties and provides spatial filtering capabilities.

\subsection{Entity Extraction}
\textbf{Class:} \texttt{ReadEntities}

This class performs a linear scan of the input file to build the core mesh database:
\begin{itemize}
    \item \textbf{Node Parsing:} Reads \texttt{*NODE} blocks to populate \texttt{Node} objects with spatial coordinates $(x,y,z)$.
    \item \textbf{Element Parsing:} Reads \texttt{*ELEMENT} blocks. It extracts the element type (e.g., C3D8R) and the connectivity list (the sequence of nodes defining the element).
    \item \textbf{Set Extraction:} The \texttt{read\_nset} method parses \texttt{*NSET} definitions, allowing the system to identify subsets of nodes referenced by boundary conditions or output requests.
\end{itemize}

\subsection{Physical Property Mapping}
\textbf{Class:} \texttt{SectionReader}

Abaqus defines properties (like thickness or material assignment) on "Element Sets", not on individual elements. This class resolves this indirection:
\begin{enumerate}
    \item \textbf{Section Discovery:} It scans for \texttt{*SOLID SECTION} and \texttt{*SHELL SECTION} keywords to identify which Element Sets define the physics of the model.
    \item \textbf{Property Extraction:} For shells, it extracts the defined thickness and number of integration points.
    \item \textbf{Element-Level Mapping:} It iterates through the model's \texttt{*ELSET} definitions to create a direct map:
    \begin{equation}
        \text{Map}(ElementID) \rightarrow \text{SectionProperties}
    \end{equation}
    This is critical for stress integration, as the software needs to know if an element is a thin shell (requires thickness integration) or a solid (centroid only).
\end{enumerate}

\subsection{Spatial Filtering}
\textbf{Classes:} \texttt{RegionFilter}, \texttt{RegionElementExtractor}

For localized analysis (e.g., analyzing only the weld bead region), processing the entire mesh is inefficient. These classes implement a bounding-box filter:

\begin{itemize}
    \item \textbf{Centroid Calculation:} For every element, the geometric center is computed based on its constituent nodes.
    \item \textbf{Box Logic:} Elements are retained only if their centroid falls within the specified window:
    \begin{equation}
        x_{min} \leq c_x \leq x_{max} \quad \text{and} \quad y_{min} \leq c_y \leq y_{max}
    \end{equation}
    \item \textbf{Type Filtering:} Optionally filters by element type (e.g., keeping only "C3D" elements to ignore 2D dummy elements).
\end{itemize}
\section{Execution Wrappers}
\label{sec:inp_execution}

\textbf{Script:} \texttt{runners.py}

This module bridges the gap between the Python data structures and the external Abaqus solver. It abstracts the complexity of command-line invocation, process management, and batch reporting.

\subsection{Single Job Execution}
\textbf{Class:} \texttt{AbaqusJobRunner}

This class handles the execution of a single simulation job. It is designed to be robust on Windows environments where path handling can be problematic.
\begin{itemize}
    \item \textbf{Command Construction:} 
    The method \texttt{\_build\_command\_string} constructs the exact CLI string required by the Abaqus driver. It handles:
    \begin{lstlisting}[language=Bash]
    abaqus job=JobName input="Path With Spaces.inp" cpus=4 memory=90 scratch="C:/Temp"
    \end{lstlisting}
    It ensures all paths are correctly quoted to prevent errors.
    
    \item \textbf{Resource Management:} 
    It translates the configuration object (from \texttt{dataclasses.py}) into solver flags, setting CPU affinity and memory limits dynamically.
    
    \item \textbf{Scratch Handling:} 
    If enabled, it automatically creates and assigns a scratch directory for temporary solver files, keeping the main output directory clean.
\end{itemize}

\subsection{Batch Orchestration}
\textbf{Class:} \texttt{INPRunner}

This class manages the execution of multiple input files in sequence.
\begin{itemize}
    \item \textbf{Process Monitoring:} It launches the Abaqus process and monitors its return code. A code of \texttt{0} implies success, while anything else triggers error logging.
    \item \textbf{Reporting:} After the batch completes, it generates a \texttt{summary\_report.txt} containing execution times and status (OK/ERROR) for every file processed. This allows the user to quickly identify failed simulations in large batches.
\end{itemize}