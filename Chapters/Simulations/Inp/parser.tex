\section{Robust INP Parsing}
\label{sec:inp_parsing}

\textbf{Script:} \texttt{parser.py}

Abaqus Input Files (\texttt{.inp}) are ASCII-based but often contain inconsistent formatting (variations in capitalization, whitespace, and parameter ordering). The \texttt{INPParser} class provides static utility methods to handle this variability robustly.

\subsection{Key Methods}
\begin{itemize}
    \item \textbf{\texttt{is\_header(line, keyword)}:} 
    Performs a case-insensitive check to see if a line starts with a specific keyword (e.g., \texttt{*ELEMENT}). This centralizes the logic for identifying Abaqus command blocks.
    
    \item \textbf{\texttt{get\_parameter(line, key)}:} 
    Extracts values from comma-separated key-value pairs typical of Abaqus headers.
    \begin{lstlisting}
    Input: "*ELEMENT, TYPE=C3D8R, ELSET=Set-1"
    Call: get_parameter(line, "ELSET")
    Output: "Set-1"
    \end{lstlisting}
    It handles edge cases like spaces around the equals sign or mixed casing, ensuring reliable metadata extraction.
\end{itemize}