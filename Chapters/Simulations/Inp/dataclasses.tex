\section{Data Structures and Configuration}
\label{sec:inp_structures}

\textbf{Script:} \texttt{dataclasses.py}

To decouple the Python logic from the specific formatting of Abaqus input files, this module defines strongly-typed data structures (using Python's \texttt{@dataclass}). These classes act as intermediate representations for the physical entities and execution configurations.

\subsection{Finite Element Entities}
\begin{itemize}
    \item \textbf{Node:} Represents a geometric point in 3D space with an ID (label) and coordinates $(x, y, z)$.
    \item \textbf{Element:} Stores the mesh topology, linking an ID (label) to a list of constituent \texttt{Node} IDs and defining the element type (e.g., C3D8R).
    \item \textbf{SectionProperties:} Abstract container for physical properties associated with element sets. It distinguishes between \texttt{Solid} (homogeneous) and \texttt{Shell} sections, storing critical thickness parameters needed for stress calculation integration points.
\end{itemize}

\subsection{Execution Configuration}
\textbf{Class:} \texttt{AbaqusJobConfig}

This class encapsulates all parameters required to launch an Abaqus solver job via the command line. It includes:
\begin{itemize}
    \item \textbf{Resource Allocation:} Number of CPUs (\texttt{n\_cpus}) and memory percentage (\texttt{memory}).
    \item \textbf{Environment Settings:} Path to the Abaqus executable (\texttt{abaqus\_cmd}) and scratch directory usage (\texttt{use\_scratch}).
    \item \textbf{Job Control:} Timeout limits (default 30 hours) to prevent stalled processes from hanging the pipeline indefinitely.
    \item \textbf{Output Mode:} \texttt{silent\_mode} — if \texttt{True}, captures stdout/stderr to a log file; if \texttt{False}, streams output live. \texttt{auto\_cleanup} — if \texttt{True}, automatically removes the scratch directory after a successful run.
\end{itemize}

\textbf{Class:} \texttt{AbaqusScriptConfig}

A companion dataclass for running standalone Abaqus Python scripts (not full solver jobs). Used by \texttt{AbaqusScriptRunner}:
\begin{itemize}
    \item \texttt{script\_name} — path to the \texttt{.py} script to execute.
    \item \texttt{working\_dir} — working directory; UNC paths (\texttt{\textbackslash\textbackslash server\textbackslash share}) are handled transparently via \texttt{pushd}/\texttt{popd}.
    \item \texttt{python\_cmd} — Python interpreter or Abaqus Python wrapper command.
    \item \texttt{env} — optional dictionary of environment variables passed to the subprocess.
\end{itemize}