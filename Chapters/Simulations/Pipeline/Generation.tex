\section{Data Standardization}
\label{sec:pl_dataclass}

\textbf{Script:} \texttt{dataclass.py}

To maintain robustness across the pipeline, raw dictionary data loaded from JSON is immediately converted into a strictly-typed object. The \texttt{SimulationConfig} dataclass acts as the central contract for the entire simulation workflow.

\subsection{Structure}
The class groups parameters into logical domains:
\begin{itemize}
    \item \textbf{Paths:} Validated \texttt{Path} objects for working directories and scripts (e.g., \texttt{geometry\_script}, \texttt{polynomial\_json\_dir}, \texttt{polynomial\_json\_default}).
    \item \textbf{Physics:} Material properties like Elastic Modulus and Poisson's Ratio.
    \item \textbf{Meshing:} Range definitions (\texttt{min}, \texttt{max}, \texttt{step}) for Design of Experiments (DOE).
    \item \textbf{Solver:} Abaqus-specific flags such as \texttt{nlgeom} (Non-Linear Geometry) and time incrementation limits.
    \item \textbf{Internal defaults:} \texttt{instance\_name}, \texttt{step\_name}, \texttt{nset\_disp\_name}, and \texttt{abaqus\_cmd} have sensible defaults and rarely need to be changed from the JSON.
\end{itemize}

\section{Automated Case Generation}
\label{sec:pl_generator}

\textbf{Script:} \texttt{generator.py}

This module implements the "Design of Experiments" (DOE) logic, automatically generating the necessary input files for a parametric study.

\subsection{Parameter Combination}
\textbf{Class:} \texttt{ParameterGenerator}

The method \texttt{generate\_combinations} takes the ranges defined in the configuration (e.g., Mesh Size from 0.6 to 1.0, Length from 50 to 100) and produces a comprehensive list of all permutation dictionaries. It generates a unique \texttt{simulation\_id} for each case (e.g., \texttt{Mesh-0\_8--Length-50}) to serve as a key throughout the pipeline.

\subsection{Geometry Fabrication}
\textbf{Class:} \texttt{GeometryGenerator}

Once parameters are defined, this class instantiates the physical models:
\begin{enumerate}
    \item \textbf{Environment Setup:} It serializes the parameter dictionary into a JSON string and injects it into the OS environment variables (\texttt{SIMULATION\_PARAMETERS}).
    \item \textbf{Abaqus CAE Execution:} It invokes the configured Abaqus Python script in \texttt{noGUI} mode. This script reads the environment variables and constructs the \texttt{.inp} file programmatically.
    \item \textbf{Batch Management:} It iterates through the entire list of combinations, ensuring a dedicated directory is created and populated for every simulation case.
\end{enumerate}