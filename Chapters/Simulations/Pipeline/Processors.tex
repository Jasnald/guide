\section{Simulation Processors (Business Logic)}
\label{sec:pl_processors}

\textbf{Script:} \texttt{processors.py}

\begin{quote}
    \textbf{Development Note:} This module is currently identified as the primary candidate for future refactoring. As new simulation types are added, the logic here tends to grow in complexity, and a move towards a more polymorphic architecture (e.g., specific Strategy classes for CM vs. RSA) is planned.
\end{quote}

This script implements the specific "business logic" for modifying input files based on the experiment type. It acts as the glue between the static configuration (\texttt{config.py}) and the low-level manipulation tools (\texttt{\_inp\_modules}).

\subsection{Contour Method Processor}
\textbf{Class:} \texttt{ContourProcessor}

This class manages the application of boundary conditions derived from the surface measurements (Module A/B).

\begin{itemize}
    \item \textbf{Data Ingestion:} 
    It loads the polynomial coefficients (JSON) representing the cut surface. It supports both "Batch Mode" (matching JSONs to samples by name) and "Single Mode" (applying one reference surface to all simulations for sensitivity analysis).

    \item \textbf{Nodal Displacement Calculation:} 
    For every node on the cut surface (identified by \texttt{nset\_disp\_name}), it evaluates the polynomial $Z = P(x, y)$ using the imported \texttt{calculate\_z\_polynomial} function. This translates the experimental roughness into simulation boundary conditions.
\end{itemize}

\subsection{Residual Stress Processor}
\textbf{Class:} \texttt{ResidualStressProcessor} (and variants)

\textit{Note: While sharing the same file, this logic handles the mapping of stress fields.}

\begin{itemize}
    \item \textbf{CSV Matching:} 
    It implements a heuristic to pair Abaqus input files with their corresponding stress data (CSV/Dataframes) generated by the ElementProcess module. It attempts exact name matching first, falling back to partial containment matching if necessary.

    \item \textbf{Injection Workflow:} 
    Once paired, it uses the \texttt{\_inp\_modules} to:
    \begin{enumerate}
        \item Generate \texttt{*Elset} definitions for the elements receiving stress.
        \item Create the \texttt{*Initial Conditions, type=STRESS} block.
        \item Update material properties (Elastic/Plastic) based on the global configuration, ensuring the simulation runs with the correct physical parameters.
    \end{enumerate}
\end{itemize}