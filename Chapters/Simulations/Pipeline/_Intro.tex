\chapter{Automation Pipeline Core}
\label{ch:pipeline_core}

\textbf{Module Path:} \texttt{src/simulations/pipeline/}

This module functions as the high-level orchestration layer for the entire simulation framework. While individual scripts (like \texttt{cm\_main.py}) define the specific physics of an experiment, the \texttt{pipeline} module handles the "logistics": directory management, configuration loading, tool chaining, and execution flow control.

\textbf{Architectural Stability:} This module is designed as a closed, stable framework. It provides the standardized infrastructure upon which specific simulation scripts are built. \textbf{Users are typically not expected to modify these files}, as changes here affect the global behavior of all simulation types.

\section{Core Responsibilities}

The pipeline abstracts repetitive tasks into reusable components, ensuring consistency across different experimental modules (Module A, Module B, etc.).

\begin{itemize}
    \item \textbf{Configuration Management (\texttt{config.py}):} 
    Centralizes the loading and validation of the global \texttt{config.json}. It ensures that all paths, parameters, and flags are correctly propagated to the solvers.
    
    \item \textbf{Process Abstraction (\texttt{processors.py}, \texttt{converters.py}):} 
    Wraps the lower-level tools (such as the Element Extractor or the ODB Converter) into unified Python classes. This allows the main simulation scripts to call complex operations via simple methods like \texttt{processor.run()}.
    
    \item \textbf{Workspace Management (\texttt{generator.py}, \texttt{clear\_dir.py}):} 
    Automates the creation of standardized directory structures (Input/Output/Post) and handles cleanup tasks, ensuring a pristine environment for each simulation batch.
    
    \item \textbf{Data Standardization (\texttt{dataclass.py}):} 
    Defines strict data contracts for passing information between stages, reducing the risk of type errors or missing parameters during complex multi-step simulations.
\end{itemize}

\section{Configuration Management}
\label{sec:pl_config}

\textbf{Script:} \texttt{config.py}

The \texttt{ConfigurationManager} class serves as the single source of truth for the simulation parameters. It decouples the Python logic from the user inputs, loading settings from an external \texttt{config.json} file and populating a strictly-typed \texttt{SimulationConfig} object.

\subsection{Loading Logic}
The \texttt{load()} method implements a "fail-safe" loading strategy:
\begin{enumerate}
    \item \textbf{JSON Parsing:} Reads the raw JSON structure.
    \item \textbf{Path Resolution:} Resolves relative paths (e.g., \texttt{./CM\_Simulations}) to absolute paths based on the project root, ensuring portability across different machines.
    \item \textbf{Default Fallbacks:} If specific parameters (like \texttt{mesh\_step} or \texttt{n\_cpus}) are missing from the JSON, the manager assigns hardcoded default values (e.g., $E=210000$ MPa, $\nu=0.3$) to guarantee the simulation can proceed.
\end{enumerate}

\section{Workspace Hygiene}
\label{sec:pl_cleardir}

\textbf{Script:} \texttt{clear\_dir.py}

To prevent data contamination between simulation runs—where results from a previous iteration might be mistakenly read by the current job—this script provides a robust cleaning utility.

\subsection{Functionality}
The function \texttt{ClearDirectory(target\_dir)} performs a deep clean:
\begin{itemize}
    \item \textbf{Validation:} Checks if the target directory exists to avoid errors.
    \item \textbf{Recursive Removal:} Iterates through the directory contents, distinguishing between files (removed via \texttt{os.unlink}) and subdirectories (removed via \texttt{shutil.rmtree}).
    \item \textbf{Error Handling:} Wraps deletions in try-except blocks to report specific file access errors without crashing the entire pipeline.
\end{itemize}
\section{Results Conversion Pipeline}
\label{sec:pl_conversion}

\textbf{Script:} \texttt{converters.py}

While the \texttt{Conversor} module (Part I) contains the low-level logic for handling data formats, this script acts as the high-level trigger within the automation pipeline. It orchestrates the multi-stage process of transforming proprietary Abaqus results into open formats.

\subsection{Two-Stage Execution}
The \texttt{ResultConverter} class manages the bridge between the different Python environments required for extraction:

\begin{enumerate}
    \item \textbf{Abaqus Extraction (Python 2.7):} 
    The method \texttt{\_run\_abaqus\_extraction} triggers the \texttt{ODB\_2\_XDMF.py} script using the configured Abaqus command line. It wraps the call in an \texttt{AbaqusScriptRunner}, ensuring the proprietary ODB API is accessed correctly to dump raw data into NPY files.
    
    \item \textbf{XDMF Compilation (Python 3.x):} 
    Immediately after extraction, the method \texttt{\_run\_npy\_to\_xdmf} invokes the \texttt{NpyBatchToXdmfConverter}. Since this runs in the modern pipeline environment, it efficiently compiles the raw arrays into hierarchical HDF5 files ready for visualization.
\end{enumerate}

\subsection{Pipeline Integration}
This converter is designed to be called at the end of a simulation workflow (e.g., inside \texttt{cm\_main.py}). It takes a \texttt{method\_type} (like "Contour Method") as an argument to correctly route the output files to their specific directories.
\section{Data Standardization}
\label{sec:pl_dataclass}

\textbf{Script:} \texttt{dataclass.py}

To maintain robustness across the pipeline, raw dictionary data loaded from JSON is immediately converted into a strictly-typed object. The \texttt{SimulationConfig} dataclass acts as the central contract for the entire simulation workflow.

\subsection{Structure}
The class groups parameters into logical domains:
\begin{itemize}
    \item \textbf{Paths:} Validated \texttt{Path} objects for working directories and scripts (e.g., \texttt{geometry\_script}, \texttt{polynomial\_json\_dir}, \texttt{polynomial\_json\_default}).
    \item \textbf{Physics:} Material properties like Elastic Modulus and Poisson's Ratio.
    \item \textbf{Meshing:} Range definitions (\texttt{min}, \texttt{max}, \texttt{step}) for Design of Experiments (DOE).
    \item \textbf{Solver:} Abaqus-specific flags such as \texttt{nlgeom} (Non-Linear Geometry) and time incrementation limits.
    \item \textbf{Internal defaults:} \texttt{instance\_name}, \texttt{step\_name}, \texttt{nset\_disp\_name}, and \texttt{abaqus\_cmd} have sensible defaults and rarely need to be changed from the JSON.
\end{itemize}

\section{Automated Case Generation}
\label{sec:pl_generator}

\textbf{Script:} \texttt{generator.py}

This module implements the "Design of Experiments" (DOE) logic, automatically generating the necessary input files for a parametric study.

\subsection{Parameter Combination}
\textbf{Class:} \texttt{ParameterGenerator}

The method \texttt{generate\_combinations} takes the ranges defined in the configuration (e.g., Mesh Size from 0.6 to 1.0, Length from 50 to 100) and produces a comprehensive list of all permutation dictionaries. It generates a unique \texttt{simulation\_id} for each case (e.g., \texttt{Mesh-0\_8--Length-50}) to serve as a key throughout the pipeline.

\subsection{Geometry Fabrication}
\textbf{Class:} \texttt{GeometryGenerator}

Once parameters are defined, this class instantiates the physical models:
\begin{enumerate}
    \item \textbf{Environment Setup:} It serializes the parameter dictionary into a JSON string and injects it into the OS environment variables (\texttt{SIMULATION\_PARAMETERS}).
    \item \textbf{Abaqus CAE Execution:} It invokes the configured Abaqus Python script in \texttt{noGUI} mode. This script reads the environment variables and constructs the \texttt{.inp} file programmatically.
    \item \textbf{Batch Management:} It iterates through the entire list of combinations, ensuring a dedicated directory is created and populated for every simulation case.
\end{enumerate}
\section{Simulation Processors (Business Logic)}
\label{sec:pl_processors}

\textbf{Script:} \texttt{processors.py}

\begin{quote}
    \textbf{Development Note:} This module is currently identified as the primary candidate for future refactoring. As new simulation types are added, the logic here tends to grow in complexity, and a move towards a more polymorphic architecture (e.g., specific Strategy classes for CM vs. RSA) is planned.
\end{quote}

This script implements the specific "business logic" for modifying input files based on the experiment type. It acts as the glue between the static configuration (\texttt{config.py}) and the low-level manipulation tools (\texttt{\_inp\_modules}).

\subsection{Contour Method Processor}
\textbf{Class:} \texttt{ContourProcessor}

This class manages the application of boundary conditions derived from the surface measurements (Module A/B).

\begin{itemize}
    \item \textbf{Data Ingestion:} 
    It loads the polynomial coefficients (JSON) representing the cut surface. It supports both "Batch Mode" (matching JSONs to samples by name) and "Single Mode" (applying one reference surface to all simulations for sensitivity analysis).

    \item \textbf{Nodal Displacement Calculation:} 
    For every node on the cut surface (identified by \texttt{nset\_disp\_name}), it evaluates the polynomial $Z = P(x, y)$ using the imported \texttt{calculate\_z\_polynomial} function. This translates the experimental roughness into simulation boundary conditions.
\end{itemize}

\subsection{Residual Stress Processor}
\textbf{Class:} \texttt{ResidualStressProcessor} (and variants)

\textit{Note: While sharing the same file, this logic handles the mapping of stress fields.}

\begin{itemize}
    \item \textbf{CSV Matching:} 
    It implements a heuristic to pair Abaqus input files with their corresponding stress data (CSV/Dataframes) generated by the ElementProcess module. It attempts exact name matching first, falling back to partial containment matching if necessary.

    \item \textbf{Injection Workflow:} 
    Once paired, it uses the \texttt{\_inp\_modules} to:
    \begin{enumerate}
        \item Generate \texttt{*Elset} definitions for the elements receiving stress.
        \item Create the \texttt{*Initial Conditions, type=STRESS} block.
        \item Update material properties (Elastic/Plastic) based on the global configuration, ensuring the simulation runs with the correct physical parameters.
    \end{enumerate}
\end{itemize}