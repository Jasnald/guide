\chapter{Abaqus Scripting Interface (ASI) Framework}
\label{ch:abaqus_modules}

\textbf{Module Path:} \texttt{src/simulations/\_modules/}

\begin{quote}
    \textbf{Architecture Note:} This module represents the deepest layer of the simulation infrastructure. Unlike the text-based manipulation of \texttt{inp\_process}, this framework operates directly within the Abaqus kernel (Python 2.7), utilizing the official API (\texttt{mdb}, \texttt{part}, \texttt{assembly}) to construct models programmatically.
\end{quote}

\section{Overview}

This library provides a modular, object-oriented abstraction over the verbose Abaqus scripting interface. It is organized into functional domains to separate geometry generation, property assignment, and meshing logic.

\subsection{Core Components}
The framework is divided into four primary sub-packages:

\begin{itemize}
    \item \textbf{Core (\texttt{\_modules/core}):} 
    Handles the standard Finite Element definitions common to all simulations. This includes Job creation, Mesh control (seeding, element types like C3D8R), and Step definitions.
    
    \item \textbf{Assignment (\texttt{\_modules/assigment}):} 
    Manages the physical properties of the model. It contains dedicated modules for Material definition, Section creation, and the instantiation of parts into the assembly.
    
    \item \textbf{Geometry (\texttt{\_modules/geometry}):} 
    Implements the "CAD" logic. It uses a Strategy pattern where each simulation type (e.g., \texttt{sim\_one} for T-Shape, \texttt{sim\_two} for Milling) has its own geometry generator, ensuring that new experiments can be added without modifying the core logic.
    
    \item \textbf{Setup \& Utilities (\texttt{\_modules/geometry\_setup}, \texttt{utilitary}):} 
    Provides high-level helpers for partitioning, datum plane creation, and boundary conditions. The utility module includes \textbf{Mixins} for logging and context management, allowing distinct classes to share common functionalities seamlessly.
\end{itemize}

\section{Design Philosophy}

The module uses a \textbf{Setter-based service architecture} centred on three building blocks:

\begin{itemize}
    \item \textbf{\texttt{ContextMixin}:} Carries the shared Abaqus objects (\texttt{model}, \texttt{t\_part}, \texttt{instance\_name}, mesh size, length, etc.) and exposes \texttt{bind\_context} / \texttt{propagate\_to} to push those values down the full object graph automatically.

    \item \textbf{\texttt{LoggerMixin}:} Inherits from \texttt{ContextMixin} and attaches a per-class \texttt{setup\_logger} instance on construction. All \texttt{Setter} classes inherit from this mixin.

    \item \textbf{\texttt{ServiceMixin} + \texttt{SERVICE\_CATALOG}:} \texttt{ServiceMixin.run(catalog)} iterates a dictionary that maps attribute names (e.g., \texttt{\_mat}, \texttt{\_mesh}, \texttt{\_job}) to their \texttt{Setter} classes. It instantiates each one, binds it to \texttt{self}, and propagates context to all of them in one call.
\end{itemize}

The concrete simulation classes inherit from \textbf{\texttt{BaseAnalysis(LoggerMixin, ServiceMixin)}}. The constructor calls \texttt{ServiceMixin.run(SERVICE\_CATALOG)}, which builds all services, and then \texttt{bind\_context}, which wires the shared Abaqus objects into every service. Subclasses need only implement four abstract methods (\texttt{\_get\_geometry\_setter}, \texttt{\_setup\_steps}, \texttt{\_setup\_boundary\_conditions}, \texttt{\_setup\_partitions}); \texttt{run\_analysis()} then drives the full template sequence.

\section{Base Class}
\label{sec:am_base}

\textbf{Script:} \texttt{base.py} \quad \textbf{Class:} \texttt{BaseAnalysis(LoggerMixin, ServiceMixin)}

This is the single entry point for every concrete simulation script. It wires geometry, services, and the execution template together.

\subsection{Constructor}

\texttt{\_\_init\_\_(params, output\_dir)} performs four sequential steps:

\begin{enumerate}
    \item \textbf{Parameter unpacking:} Iterates \texttt{\_PARAM\_KEYS} and calls \texttt{setattr} to map each JSON key to the corresponding Python attribute:
    \begin{table}[ht]
    \centering
    \begin{tabular}{ll}
    \hline
    \textbf{JSON key} & \textbf{Python attribute} \\
    \hline
    \texttt{mesh\_size}  & \texttt{self.mesh\_size} \\
    \texttt{length}      & \texttt{self.comprimento} \\
    \texttt{initialInc}  & \texttt{self.initialInc} \\
    \texttt{maxInc}      & \texttt{self.maxInc} \\
    \texttt{maxNumInc}   & \texttt{self.maxNumInc} \\
    \texttt{minInc}      & \texttt{self.minInc} \\
    \texttt{nlgeom}      & \texttt{self.nlgeom} \\
    \texttt{time}        & \texttt{self.time} \\
    \hline
    \end{tabular}
    \caption{\texttt{\_PARAM\_KEYS} mapping in \texttt{BaseAnalysis}.}
    \end{table}

    \item \textbf{Geometry creation:} Calls the abstract \texttt{\_get\_geometry\_setter()} (implemented by the subclass) and immediately calls \texttt{\_geometry(depth=self.comprimento)}, which returns the \texttt{(model, t\_part)} tuple and sets \texttt{self.instance\_name = t\_part.name + "-1"}.

    \item \textbf{Service initialisation:} \texttt{\_init\_services()} imports \texttt{SERVICE\_CATALOG} from \texttt{\_modules}, calls \texttt{ServiceMixin.run(catalog)} to instantiate and bind all setters (e.g.\ \texttt{self.\_mat}, \texttt{self.\_sec}, \texttt{self.\_inst}, \texttt{self.\_mesh}, \texttt{self.\_job}), then calls \texttt{bind\_context} to propagate \texttt{model}, \texttt{t\_part}, and \texttt{instance\_name} to every service.
\end{enumerate}

\subsection{Template Method: \texttt{run\_analysis()}}

The public API for running a full simulation. Calls the following sequence regardless of subclass:

\begin{enumerate}
    \item \texttt{first\_stage()} --- creates material, section, and assembly instance.
    \item \texttt{\_setup\_partitions()} --- abstract; subclass applies datum planes and mesh partitions.
    \item \texttt{self.\_mesh.mesh()} --- generates the C3D8R mesh.
    \item \texttt{\_setup\_steps()} --- abstract; subclass creates analysis steps.
    \item \texttt{\_setup\_boundary\_conditions()} --- abstract; subclass applies loads and BCs.
    \item \texttt{self.\_job.create\_and\_move\_job()} --- writes the \texttt{.inp} and moves it to \texttt{output\_dir}.
\end{enumerate}

\begin{remark}
\texttt{first\_stage()} hard-codes $E = 210\,000$ MPa and $\nu = 0.3$. These values are overwritten later by \texttt{INPInserter.replace\_material\_block} in the \texttt{pipeline} module using the values from \texttt{config.json}, so the hard-coded defaults only matter if the pipeline step is skipped.
\end{remark}

\subsection{Abstract Methods (Subclass Contract)}

\begin{table}[ht]
\centering
\begin{tabular}{ll}
\hline
\textbf{Method} & \textbf{Responsibility} \\
\hline
\texttt{\_get\_geometry\_setter()} & Return the correct \texttt{GeometrySetter} instance. \\
\texttt{\_setup\_steps()}          & Create Abaqus analysis steps. \\
\texttt{\_setup\_boundary\_conditions()} & Apply loads, BCs, and constraints. \\
\texttt{\_setup\_partitions()}     & Apply datum planes and mesh partitions. \\
\hline
\end{tabular}
\caption{Abstract methods that every concrete simulation class must implement.}
\end{table}

\section{Core Definitions (Job, Step, Mesh)}
\label{sec:am_core}

\textbf{Package:} \texttt{\_modules/core/}

This package establishes the fundamental Finite Element Analysis (FEA) settings that are consistent across different simulation types. It abstracts the standard Abaqus commands for job submission, time-stepping, and discretization into reusable classes.

\subsection{Job Management}
\textbf{Script:} \texttt{\_set\_job.py}

The \texttt{JobSetter} class generates the Abaqus job and delivers the final \texttt{.inp} file to the pipeline output directory.
\begin{itemize}
    \item \textbf{Naming:} The job name is derived from the simulation parameters as \texttt{Mesh-\{mesh\_size\}--Length-\{comprimento\}} with dots replaced by underscores (e.g., \texttt{Mesh-0\_6--Length-50}).
    \item \textbf{Generation:} Calls \texttt{mdb.Job(name, model)} then \texttt{job.writeInput()} to produce the \texttt{.inp} file on disk.
    \item \textbf{Relocation:} Moves the generated file from the Abaqus working directory to \texttt{output\_dir}. If a file with the same name already exists at the destination it is removed first.
\end{itemize}

\subsection{Analysis Step}
\textbf{Script:} \texttt{\_set\_step.py}

\texttt{StepSetter.create()} configures the \textbf{Static, General} step.
\begin{itemize}
    \item \textbf{Idempotency:} Deletes any existing step with the same name before re-creating it, so scripts can be re-run safely.
    \item \textbf{Incrementation:} Sets \texttt{initialInc}, \texttt{maxInc}, \texttt{maxNumInc}, \texttt{minInc} from the simulation parameters; \texttt{nlgeom} flag is passed directly.
    \item \textbf{Solver:} Uses \texttt{matrixSolver=ITERATIVE}.
    \item \textbf{Output requests:} Default \texttt{F-Output-1} and \texttt{H-Output-1} are deleted and replaced with a single custom \texttt{FieldOutputRequest} restricted to \texttt{S} (stress) and \texttt{U} (displacement) only, keeping ODB file sizes small.
\end{itemize}

\subsection{Mesh Strategy}
\textbf{Script:} \texttt{\_set\_mesh.py} (and submodules)

All meshing logic is consolidated in \texttt{MeshSetter.mesh()}, which executes four sequential steps:
\begin{enumerate}
    \item \textbf{Delete existing mesh:} Any previous mesh is removed via \texttt{deleteMesh} before re-meshing (errors are caught silently).
    \item \textbf{Global seeding:} \texttt{seedPart(size=mesh\_size)} applies the DOE mesh size with \texttt{deviationFactor=0.001}.
    \item \textbf{Element type:} \texttt{C3D8R} (8-node linear brick, reduced integration, STANDARD library) is assigned to all cells.
    \item \textbf{Generate:} \texttt{generateMesh()} finalises the discretisation.
\end{enumerate}
The submodule files (\texttt{\_set\_mesh\_bias.py}, \texttt{\_set\_mesh\_del.py}, \texttt{\_set\_mesh\_sc8r.py}, \texttt{\_set\_mesh\_seed.py}, \texttt{\_set\_mesh\_stack.py}, \texttt{\_set\_mesh\_sweep.py}) contain specialised helpers for bias seeding, sweep direction, and stack orientation that are composed into \texttt{MeshSetter} via the service catalog for more complex geometries.
\section{Physical Properties and Assembly}
\label{sec:am_assignment}

\textbf{Package:} \texttt{\_modules/assigment/} (sic)

Once the geometry is generated, it must be assigned physical properties and instantiated within the simulation assembly. This package manages the material definitions, section creations, and the hierarchical assembly process.

\subsection{Material Definition}
\textbf{Script:} \texttt{\_set\_material.py}

The \texttt{MaterialMixin} class creates the constitutive models in the Abaqus database.
\begin{itemize}
    \item \textbf{Elasticity:} Defines the \texttt{Elastic} behavior using Young's Modulus and Poisson's Ratio provided by the global configuration.
    \item \textbf{Plasticity:} Optionally adds \texttt{Plastic} behavior. This is crucial for residual stress analysis, as the redistribution of stresses often induces localized yielding. The framework checks if plastic properties are defined in the config before creating this node in the material graph.
\end{itemize}

\subsection{Section Management}
\textbf{Scripts:} \texttt{\_set\_section.py}, \texttt{\_set\_section\_assign.py}

In Abaqus, materials are referenced by "Sections", which are then assigned to geometry regions.
\begin{enumerate}
    \item \textbf{Creation:} The \texttt{SectionMixin} creates a \textbf{Solid Homogeneous Section}. It links the previously defined material (e.g., "WORK\_PIECE\_MATERIAL") to this section definition.
    \item \textbf{Assignment:} The \texttt{SectionAssignMixin} applies this section to the actual part geometry. It targets the \textbf{Whole Part} (Cells) to ensure the entire volume simulates the specified metal properties.
\end{enumerate}

\subsection{Assembly Instantiation}
\textbf{Script:} \texttt{\_set\_instance.py}

Abaqus distinguishes between "Parts" (geometry definitions) and "Instances" (occurrences in the assembly). The solver runs on the assembly.
\begin{itemize}
    \item \textbf{Instance Creation:} The \texttt{InstanceMixin} imports the generated Part into the Assembly module.
    \item \textbf{Naming Convention:} It forces the instance name to match the configuration default (e.g., \texttt{T\_SHAPE\_PART-1}). This strict naming is vital for the external \texttt{\_inp\_modules} (Part II) to correctly locate and inject boundary conditions later in the pipeline.
\end{itemize}
\section{Geometry Construction Strategies}
\label{sec:am_geometry}

\textbf{Package:} \texttt{\_modules/geometry/}

This package implements the Computer-Aided Design (CAD) logic of the framework. To support multiple experimental setups without code duplication, it employs a \textbf{Strategy Pattern}: each simulation type (e.g., T-Shape, Milling) is encapsulated in its own sub-package (\texttt{sim\_one}, \texttt{sim\_two}, etc.), but they all expose a consistent \texttt{ModelMixin} interface.

\subsection{Strategy 1: T-Shape (Sim One)}
\textbf{Sub-package:} \texttt{sim\_one/}

Designed for the standard Residual Stress benchmark (Module A), this strategy generates a parametric T-shaped beam.

\begin{itemize}
    \item \textbf{Vertex Calculation (\texttt{\_get\_shape.py}):} 
    Calculates the 2D coordinates of the T-profile cross-section based on the provided dimensions (web width, flange height, etc.). It returns a closed loop of points ensuring geometric continuity.
    
    \item \textbf{Solid Extrusion (\texttt{\_set\_geometry.py}):} 
    Uses the Abaqus API \texttt{BaseSolidExtrude}. It draws the calculated profile on a sketch plane and extrudes it by the specified \texttt{Length} parameter to create the 3D part.
\end{itemize}

\subsection{Strategy 2: Milling Profile (Sim Two)}
\textbf{Sub-package:} \texttt{sim\_two/}

Designed for the Profile Analysis experiment (Module B), dealing with material removal or complex boundary conditions.

\begin{itemize}
    \item \textbf{Geometry Logic (\texttt{\_set\_geometry2.py}):} 
    Unlike the simple extrusion of Sim One, this module may handle additional geometric features or different orientation requirements specific to the milling setup.
    
    \item \textbf{Coordinate Handling (\texttt{\_get\_shape2.py}):} 
    Provides the specialized vertex logic required for this specific profile, ensuring the mesh seeds align with the measurement points.
\end{itemize}

\subsection{Integration: The Model Mixin}
\textbf{Files:} \texttt{model\_mixin.py} (in each sub-package)

The \texttt{ModelMixin} class acts as the standardized connector. Regardless of whether the underlying geometry is a T-Shape or a Milling plate, this mixin provides the main \texttt{build\_model()} method that the orchestration script calls. This allows the high-level pipeline to switch between experiments simply by importing a different Mixin, without changing the execution logic.