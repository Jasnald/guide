\section{Core Definitions (Job, Step, Mesh)}
\label{sec:am_core}

\textbf{Package:} \texttt{\_modules/core/}

This package establishes the fundamental Finite Element Analysis (FEA) settings that are consistent across different simulation types. It abstracts the standard Abaqus commands for job submission, time-stepping, and discretization into reusable classes.

\subsection{Job Management}
\textbf{Script:} \texttt{\_set\_job.py}

The \texttt{JobSetter} class generates the Abaqus job and delivers the final \texttt{.inp} file to the pipeline output directory.
\begin{itemize}
    \item \textbf{Naming:} The job name is derived from the simulation parameters as \texttt{Mesh-\{mesh\_size\}--Length-\{comprimento\}} with dots replaced by underscores (e.g., \texttt{Mesh-0\_6--Length-50}).
    \item \textbf{Generation:} Calls \texttt{mdb.Job(name, model)} then \texttt{job.writeInput()} to produce the \texttt{.inp} file on disk.
    \item \textbf{Relocation:} Moves the generated file from the Abaqus working directory to \texttt{output\_dir}. If a file with the same name already exists at the destination it is removed first.
\end{itemize}

\subsection{Analysis Step}
\textbf{Script:} \texttt{\_set\_step.py}

\texttt{StepSetter.create()} configures the \textbf{Static, General} step.
\begin{itemize}
    \item \textbf{Idempotency:} Deletes any existing step with the same name before re-creating it, so scripts can be re-run safely.
    \item \textbf{Incrementation:} Sets \texttt{initialInc}, \texttt{maxInc}, \texttt{maxNumInc}, \texttt{minInc} from the simulation parameters; \texttt{nlgeom} flag is passed directly.
    \item \textbf{Solver:} Uses \texttt{matrixSolver=ITERATIVE}.
    \item \textbf{Output requests:} Default \texttt{F-Output-1} and \texttt{H-Output-1} are deleted and replaced with a single custom \texttt{FieldOutputRequest} restricted to \texttt{S} (stress) and \texttt{U} (displacement) only, keeping ODB file sizes small.
\end{itemize}

\subsection{Mesh Strategy}
\textbf{Script:} \texttt{\_set\_mesh.py} (and submodules)

All meshing logic is consolidated in \texttt{MeshSetter.mesh()}, which executes four sequential steps:
\begin{enumerate}
    \item \textbf{Delete existing mesh:} Any previous mesh is removed via \texttt{deleteMesh} before re-meshing (errors are caught silently).
    \item \textbf{Global seeding:} \texttt{seedPart(size=mesh\_size)} applies the DOE mesh size with \texttt{deviationFactor=0.001}.
    \item \textbf{Element type:} \texttt{C3D8R} (8-node linear brick, reduced integration, STANDARD library) is assigned to all cells.
    \item \textbf{Generate:} \texttt{generateMesh()} finalises the discretisation.
\end{enumerate}
The submodule files (\texttt{\_set\_mesh\_bias.py}, \texttt{\_set\_mesh\_del.py}, \texttt{\_set\_mesh\_sc8r.py}, \texttt{\_set\_mesh\_seed.py}, \texttt{\_set\_mesh\_stack.py}, \texttt{\_set\_mesh\_sweep.py}) contain specialised helpers for bias seeding, sweep direction, and stack orientation that are composed into \texttt{MeshSetter} via the service catalog for more complex geometries.