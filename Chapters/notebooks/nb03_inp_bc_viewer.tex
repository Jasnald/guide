\chapter{\texttt{inp\_bc\_viewer.ipynb} --- INP Boundary-Condition Visualiser}
\label{ch:nb_inp_bc}

\textbf{File:} \texttt{notebook/inp\_bc\_viewer.ipynb}\\
\textbf{Dependencies:} \texttt{numpy}, \texttt{matplotlib}, \texttt{scipy.interpolate.griddata}, \texttt{re} (standard library only)

\section{Purpose}

After \texttt{ResidualProcessor} injects the \texttt{*Initial Conditions} block and \texttt{ContourProcessor} writes the displacement boundary conditions into an \texttt{.inp} file, it is important to verify that the applied nodal field matches the fitted surface before submitting the job to Abaqus.

This notebook parses a generated \texttt{\_FI.inp} file, extracts the $U_3$ (DoF~3, $Z$-direction) displacement values applied at every node, and renders them as both a 2-D contour heatmap and a 3-D surface.

\section{Single-Cell Architecture}

The entire notebook consists of one large code cell divided into three logical sections:

\subsection{Section 1 --- Parser: \texttt{parse\_inp\_bc\_visualization()}}

\begin{lstlisting}[language=Python]
nodes, displacements = parse_inp_bc_visualization(inp_path)
\end{lstlisting}

A line-by-line parser that reads the \texttt{.inp} file and populates two dictionaries:

\begin{table}[ht]
\centering
\begin{tabular}{lp{9.5cm}}
\hline
\textbf{Output} & \textbf{Content} \\
\hline
\texttt{nodes} & \texttt{\{node\_id: (x, y, z)\}} --- all nodes from the \texttt{*Node} section. \\
\texttt{displacements} & \texttt{\{node\_id: value\}} --- boundary condition values filtered to DoF~3 only (from the \texttt{*Boundary} section). \\
\hline
\end{tabular}
\end{table}

The parser handles the Abaqus assembly-level naming convention (e.g.\ \texttt{T\_SHAPE\_PART-1.1234}) by splitting on \texttt{'.'} and reading only the integer part after the last dot. Comment lines (\texttt{**}) and non-numeric lines are silently skipped.

\begin{quote}
    \textbf{Parsing logic:} State flags \texttt{reading\_nodes} and \texttt{reading\_bcs} are toggled by keyword lines (\texttt{*Node}, \texttt{*Boundary}, \texttt{*Element}, or any other \texttt{*} keyword). Only DoF~3 entries are stored to displacements.
\end{quote}

\subsection{Section 2 --- Plotter: \texttt{plot\_bc\_surface()}}

Takes the node and displacement dictionaries, aligns them (only nodes that have a boundary condition are plotted), and produces two figures:

\begin{enumerate}
    \item \textbf{2-D Contour Heatmap:}
    \begin{itemize}
        \item Node positions are interpolated onto a $100 \times 100$ regular grid via \texttt{scipy.interpolate.griddata} with cubic interpolation.
        \item Rendered as a filled contour plot (\texttt{contourf}, 100 levels, \texttt{jet} colormap).
        \item Raw BC nodes are overlaid as scatter points for comparison.
        \item Colourbar label: \emph{Applied Displacement U3 (mm)}.
    \end{itemize}

    \item \textbf{3-D Surface:}
    \begin{itemize}
        \item \texttt{ax.plot\_trisurf(x, y, z)} renders the displacement field directly from the unstructured node data.
        \item Colourbar: $U_3$ (mm).
    \end{itemize}
\end{enumerate}

\subsection{Section 3 --- Execution Block}

\begin{lstlisting}[language=Python]
arquivo_inp = r"C:\Simulation4\Contour_Method\s1_Mesh-2_98--Length-10_FI.inp"
plot_bc_surface(arquivo_inp)
\end{lstlisting}

\begin{quote}
    \textbf{Where to change:} Update \texttt{arquivo\_inp} to point to the \texttt{\_FI.inp} file you want to inspect. Any file generated by \texttt{ContourProcessor} (stored in \texttt{data/output/exp*/cm\_directory/}) is compatible.
\end{quote}

\section{Expected Output}

Two Matplotlib figures, inline:
\begin{enumerate}
    \item A 2-D XY contour map of $U_3$ showing the spatial distribution of the applied displacements over the midsection cut face.
    \item A 3-D triangulated surface of the same displacement field, allowing perspective inspection of the deformation magnitude and asymmetry.
\end{enumerate}

Diagnostic counts are printed to stdout:
\begin{lstlisting}[language=Python]
Reading file: s1_Mesh-2_98--Length-10_FI.inp...
--> Nodes found: 4823
--> Boundary conditions found: 311
\end{lstlisting}
