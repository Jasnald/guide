\section{Stage 3: 1D Curve Fitting}
\label{sec:exp2_stage3}

\textbf{Script:} \texttt{s3\_Curve\_process.py}

Unlike the bivariate surface fitting in Module A, milling profiles are modeled as univariate polynomials $Z = P(x)$. This stage fits curves of varying degrees (1 to $N$) to the cleaned profile data, utilizing Ridge Regression to control overfitting.

\subsection{Regularized Fitting (Ridge Regression)}

To prevent Runge's phenomenon—where high-degree polynomials oscillate wildly at the edges—the script employs **Ridge Regression** (Tikhonov Regularization).

Instead of minimizing just the residual sum of squares ($||Ax - b||^2$), the algorithm minimizes:
\begin{equation}
    J(\beta) = ||\mathbf{X}\beta - \mathbf{y}||^2 + \alpha ||\beta||^2
\end{equation}
Where:
\begin{itemize}
    \item $\mathbf{X}$ is the Vandermonde matrix of powers $[1, x, x^2, \dots]$.
    \item $\beta$ are the polynomial coefficients.
    \item $\alpha$ (configured as \texttt{RIDGE\_ALPHA}) is the penalty term.
\end{itemize}

The solution is computed via the closed-form normal equation modified for regularization:
\begin{equation}
    \hat{\beta} = (\mathbf{X}^T \mathbf{X} + \alpha \mathbf{I})^{-1} \mathbf{X}^T \mathbf{y}
\end{equation}
In the script, the value of $\alpha$ defaults to \textbf{1.0} (set by \texttt{RIDGE\_ALPHA = 1}), providing a moderate penalty to large coefficients.

\subsection{Coordinate Normalization and Substitution}

To ensure numerical stability during the fitting process, the X-coordinates are mapped to the interval $[-1, 1]$. However, the output coefficients must describe the curve in the original physical coordinate system.

The script handles this via \textbf{Linear Substitution} using Horner's method:
\begin{enumerate}
    \item \textbf{Normalize:} Transform $x \rightarrow t$ where $t \in [-1, 1]$.
    \item \textbf{Fit:} Calculate coefficients $C_t$ for $P(t)$.
    \item \textbf{Substitute:} Map back to $x$ using the linear relationship $t = ax + b$.
    \begin{equation}
        P(x) = \sum C_{t,i} (ax + b)^i
    \end{equation}
    The function \texttt{\_compose\_linear\_substitution} performs this expansion mathematically, returning coefficients valid for raw input values.
\end{enumerate}

\subsection{Usage}

\begin{lstlisting}[language=Bash]
python s3_Curve_process.py
\end{lstlisting}

\textbf{Process Flow:}
\begin{itemize}
    \item The script searches for the most recent data file (Modified > Iteration > Average).
    \item It iterates from Degree 1 to \texttt{MAX\_DEGREE} (default: 7).
    \item For each degree, it calculates metrics (MSE, RMSE, Max Error).
    \item \textbf{Output:} JSON files for each degree (e.g., \texttt{Sample1\_Curve\_deg5.json}) and a summary file comparing all degrees.
\end{itemize}