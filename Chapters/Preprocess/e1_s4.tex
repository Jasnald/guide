\section{Stage 4: Comparative Analysis and Averaging}
\label{sec:stage4}

\textbf{Script:} \texttt{s4\_Choosed\_plane.py}

While Stage 3 fits surfaces to individual measurements, Stage 4 performs comparative analysis to isolate specific surface features and generate a representative "master" profile for the sample batch.

\subsection{How it Works}

The script executes a two-step analytical process:

\subsubsection{1. Tilt Correction via Surface Subtraction}
Measurement data often includes a systematic "tilt" due to physical alignment during scanning. To remove this while preserving surface roughness/curvature:
\begin{enumerate}
    \item The script loads two fitted surfaces of different degrees (e.g., Degree 5 and Degree 1).
    \item It computes the \textbf{Difference Surface} ($Z_{diff}$) by subtracting the lower-degree surface (tilt/form) from the higher-degree surface (roughness/waviness):
    \begin{equation}
        Z_{diff}(x,y) = Z_{high\_degree}(x,y) - Z_{low\_degree}(x,y)
    \end{equation}
    \item This operation effectively "flattens" the data relative to its general form.
\end{enumerate}

\subsubsection{2. Multi-Side Averaging}
To create a robust simulation model, the tool averages the results from multiple sides (e.g., Side 1 and Side 2) into a single definition.
\begin{itemize}
    \item Instead of averaging the raw points, the script \textbf{averages the polynomial coefficients} ($P_{avg}$) directly:
    \begin{equation}
        P_{avg} = \frac{1}{N} \sum_{i=1}^{N} P_{side\_i}
    \end{equation}
    \item It then recalculates the $Z$ coordinates for the entire grid using these averaged parameters, ensuring a mathematically consistent surface.
\end{itemize}

\subsection{Usage}

Run the analysis script from the terminal:

\begin{lstlisting}[language=Bash]
python s4_Choosed_plane.py
\end{lstlisting}

\textbf{Output:}
\begin{itemize}
    \item \textbf{Subtraction Files:} \texttt{SideX\_\_subtraction.json} (The individual corrected surfaces).
    \item \textbf{Final Average:} \texttt{Final\_Average\_\_average\_subtraction.json} (The master profile used for reconstruction).
\end{itemize}

\subsection{Configuration}

The analysis logic is defined in the \texttt{main()} function variables:

\begin{itemize}
    \item \textbf{Degrees to Compare:} 
    \begin{lstlisting}[language=Python]
    degrees_to_use = [1, 5]
    \end{lstlisting}
    This list defines which surfaces are subtracted. The logic is always \texttt{Index 1 - Index 0} (e.g., Degree 5 minus Degree 1). This is critical for determining what frequency of surface features is retained.

    \item \textbf{Sides to Process:} 
    \begin{lstlisting}[language=Python]
    sides = ["Side1", "Side2"]
    \end{lstlisting}
    Defines which measurement sets are included in the final average. If a sample has 4 measured sides, all 4 should be listed here to improve the statistical reliability of the result.
\end{itemize}