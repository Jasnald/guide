\section{Stage 4: Comparative Analysis and Averaging}
\label{sec:exp2_stage4}

\textbf{Script:} \texttt{s4\_Choosed\_curve.py}

Similar to the surface analysis in Module A, this stage isolates specific profile features (such as roughness or waviness) by removing underlying forms (tilt or curvature) and generates a statistical average of the sample batch.

\subsection{Tilt Correction via Curve Subtraction}

To isolate the desired surface frequency, the script subtracts a lower-degree polynomial (representing form/tilt) from a higher-degree polynomial (representing the profile detail).

\subsubsection{Re-fitting Mechanism}
Unlike simple algebraic subtraction of coefficients, this script employs a numerical re-fitting approach to ensure robustness:
\begin{enumerate}
    \item \textbf{Evaluation:} It generates a synthetic domain of $x$ points (default range $[-10, 10]$).
    \item \textbf{Difference Calculation:} For every $x$, it computes the difference between the two loaded polynomial models:
    \begin{equation}
        z_{diff}(x) = P_{high\_degree}(x) - P_{low\_degree}(x)
    \end{equation}
    \item \textbf{Regression:} It fits a \textbf{new polynomial} to these difference points $(x, z_{diff})$ to obtain the parameters of the corrected curve.
\end{enumerate}

\subsection{Master Profile Averaging}

Once the subtraction curves are generated for all samples, the script computes a "Master Profile" to represent the entire batch.

This is achieved by averaging the polynomial coefficients directly:
\begin{equation}
    \mathbf{C}_{avg} = \frac{1}{N} \sum_{i=1}^{N} \mathbf{C}_{i}
\end{equation}
Where $\mathbf{C}_i$ is the vector of coefficients for sample $i$. This averaged parameter set is saved as \texttt{Average\_Subtraction.json} and serves as the input for the final reconstruction stage.

\subsection{Usage and Configuration}

\begin{lstlisting}[language=Bash]
python s4_Choosed_curve.py
\end{lstlisting}

\subsubsection{Configuration}
The behavior is controlled by variables within the \texttt{main()} function:

\begin{itemize}
    \item \textbf{Degrees to Subtract:} 
    \begin{lstlisting}[language=Python]
    deg_high = 2
    deg_low = 1
    \end{lstlisting}
    Defines which polynomial orders are used. In this default configuration, the script removes linear tilt (Degree 1) from the quadratic form (Degree 2).

    \item \textbf{Sample Range:}
    The script iterates through a hardcoded range of samples (e.g., \texttt{range(1, 10)}). This loop must be adjusted to match the actual number of samples processed in previous stages.
\end{itemize}

\textbf{Output:}
\begin{itemize}
    \item \textbf{Individual Subtractions:} \texttt{SampleX\_Subtraction.json} saved in the \texttt{Choosed\_curve} directory.
    \item \textbf{Final Average:} \texttt{Average\_Subtraction.json}.
\end{itemize}