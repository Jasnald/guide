\chapter{User's Guide: Step-by-Step Tutorial}
\label{ch:user_guide}

\section{Overview}
This tutorial guides you through the complete Residual Stress Analysis workflow using the hybrid experimental-numerical framework.

\textbf{Workflow Summary:}
\begin{enumerate}
    \item \textbf{Configuration ($\sim$20 min):} Define material properties and Design of Experiments (DOE) ranges.
    \item \textbf{Contour Method Simulation ($\sim$10 min):} Reconstruct surface stresses from experimental data.
    \item \textbf{Residual Stress Analysis ($\sim$10 min):} Simulate the stress redistribution during the cutting process.
    \item \textbf{Visualization:} Analyze results using ParaView.
\end{enumerate}

\begin{quote}
    \textbf{Note:} These time values are done considered one simulation with a mesh after the convergence of the result, which means 50 thousand elements.
\end{quote}

\textbf{Expected Outcome:} By the end of this guide, you will have XDMF/HDF5 files representing the full 3D stress tensor field, ready for post-processing.

\section{Prerequisites}
Before proceeding, ensure the following environment is set up:
\begin{itemize}
    \item \textbf{Abaqus 2021 or 2023:} Must be installed. The command \texttt{abq2021} (or equivalent) must be accessible via terminal or defined in the config.
    \item \textbf{Python 3.10+:} Required for the pipeline orchestration.
    \item \textbf{Dependencies:} Install required packages:
    \begin{lstlisting}[language=bash]
    pip install -r requirements.txt
    \end{lstlisting}
    \item \textbf{Project Structure:} Ensure the \texttt{src/} folder matches the architecture defined in Chapter \ref{ch:introduction}.
\end{itemize}