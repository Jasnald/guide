\chapter{User's Guide: Step-by-Step Tutorial}
\label{ch:user_guide}

\section{Overview}
This tutorial guides you through the complete Residual Stress Analysis workflow for both Experiment~1 (T-shaped specimen) and Experiment~2 (milled block). All entry points are consolidated in the \texttt{scripts/} directory.

\textbf{Workflow Summary:}
\begin{enumerate}
    \item \textbf{Data Processing ($\sim$5 min):} Pre-process raw measurement files and fit polynomial models to the experimental surface/curve data.
    \item \textbf{Geometry Setup ($<$1 min):} Configure the cutting-plane positions using the interactive GUI and save them to \texttt{data/e\{n\}\_plane\_settings.json}.
    \item \textbf{Contour Method Simulation ($\sim$10 min):} Build meshes, apply boundary conditions, and extract ODB results.
    \item \textbf{Residual Stress Analysis ($\sim$10 min):} Map CM stresses onto the RSA mesh, inject initial conditions, and solve.
    \item \textbf{Visualization:} Inspect results using the built-in \texttt{answer\_viewer} or ParaView.
\end{enumerate}

\begin{quote}
    \textbf{Note:} Simulation times assume a mesh-converged run of approximately 50\,000 elements.
\end{quote}

\textbf{Expected Outcome:} XDMF/HDF5 files representing the full 3-D stress tensor field, ready for post-processing in the built-in viewer or ParaView.

\section{Entry-Point Scripts}

All user-facing commands live in \texttt{scripts/}:

\begin{table}[ht]
\centering
\begin{tabular}{ll}
\hline
\textbf{Script} & \textbf{Purpose} \\
\hline
\texttt{e1\_process\_run.py}    & Process Exp1 surface measurements (3-D T-profile). \\
\texttt{e2\_process\_run.py}    & Process Exp2 curve measurements (1-D milling). \\
\texttt{setup\_geometry.py}     & Launch the cutting-plane configurator GUI. \\
\texttt{e1\_simulation\_run.py} & Unified CM + RSA runner for Experiment~1. \\
\texttt{e2\_simulation\_run.py} & Unified CM + RSA runner for Experiment~2. \\
\hline
\end{tabular}
\end{table}

\section{Prerequisites}
Before proceeding, ensure the following environment is set up:
\begin{itemize}
    \item \textbf{Abaqus 2021 or 2023:} Must be installed. The command \texttt{abq2021} (or equivalent) must be accessible via terminal or defined in \texttt{data/config.json}.
    \item \textbf{Python 3.10+:} Required for all pipeline orchestration scripts.
    \item \textbf{Dependencies:} Install required packages:
    \begin{lstlisting}[language=bash]
pip install -r requirements.txt
    \end{lstlisting}
    \item \textbf{Input data:} Raw measurement files placed in \texttt{data/input/exp1/} and \texttt{data/input/exp2/}.
\end{itemize}