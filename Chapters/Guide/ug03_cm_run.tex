\section{Step 2: Contour Method (CM) Simulation}
\label{sec:guide_cm}

\textbf{Goal:} Run the inverse calculation to reconstruct boundary conditions and solve for surface stresses.

\subsection{Action: Execution}
Run the main orchestrator from the project root:

\begin{lstlisting}[language=bash]
python src/Simulations/cm_main.py
\end{lstlisting}

The terminal will display progress logs: \texttt{[INFO] Processing Case: Mesh-0\_8--Length-50...}

\subsection{Result Verification}
Since the pipeline is automated, the system ensures data consistency before saving. Upon completion, you can locate the results in the \texttt{CM\_Directory} defined in your configuration.

\textbf{Generated Outputs:}
\begin{enumerate}
    \item \textbf{Data Files:} Navigate to the \texttt{/xdmf\_hdf5\_files/} folder. You will find the converted results for each DOE case:
    \begin{itemize}
        \item \texttt{.xdmf} (Metadata for ParaView)
        \item \texttt{.h5} (Heavy data storage)
    \end{itemize}

    \item \textbf{Execution Logs:} Review \texttt{summary\_report.txt} to see the execution time for each simulation case.
\end{enumerate}