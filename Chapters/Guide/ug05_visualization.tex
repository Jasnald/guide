\section{Step 4: Visualizing Results}
\label{sec:guide_vis}

\textbf{Goal:} Interpret the results using ParaView.

\subsection{Loading Data}
\begin{enumerate}
    \item Open \textbf{ParaView}.
    \item File $\rightarrow$ Open $\rightarrow$ Navigate to your output folder (e.g., \texttt{CM\_Directory/xdmf\_hdf5\_files/}).
    \item Select a file, e.g., \texttt{Mesh-0\_8--Length-50.xdmf}.
    \item Click \textbf{"Apply"} in the left properties panel.
\end{enumerate}

\subsection{Analyzing Stresses}
\begin{enumerate}
    \item In the toolbar, locate the variable selector (usually shows "Solid Color").
    \item Change it to \textbf{"S"} (Stress Tensor).
    \item In the component selector (next to "S"), choose:
    \begin{itemize}
        \item \textbf{S11:} Longitudinal Stress (typical interest for T-shapes).
        \item \textbf{S33:} Normal Stress (to the cut plane).
        \item \textbf{Mises:} Equivalent Von Mises stress.
    \end{itemize}
\end{enumerate}

\subsection{Interpretation Hints}
\begin{itemize}
    \item \textbf{Red Zones (Positive):} Indicate Tensile Residual Stress.
    \item \textbf{Blue Zones (Negative):} Indicate Compressive Residual Stress.
    \item \textbf{Validation:} The stress perpendicular to the cut surface (S33) should be close to zero at the cut face (boundary condition enforcement).
\end{itemize}