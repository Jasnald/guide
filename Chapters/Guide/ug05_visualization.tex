\section{Step 5: Visualizing Results}
\label{sec:guide_vis}

\textbf{Goal:} Inspect the XDMF/HDF5 output using the built-in viewer or ParaView.

\subsection{Option A: Built-in Viewer (\texttt{answer\_viewer.py})}

Launch from the project root:
\begin{lstlisting}[language=bash]
python -m simulations.gui.answer_viewer
\end{lstlisting}

\begin{enumerate}
    \item \textbf{File $\rightarrow$ Open HDF5} --- select an \texttt{.h5} file from \texttt{cm\_directory} or \texttt{rea\_directory}.
    \item Use the \textbf{Model} combo to switch between DOE cases.
    \item Select \textbf{Step}, \textbf{Frame}, and \textbf{Field} (e.g., \texttt{stress\_tensor}).
    \item Click \textbf{Update Visualization} or change any combo to refresh the 3-D view.
\end{enumerate}

\textbf{Key controls:}
\begin{itemize}
    \item \textbf{View buttons (+X/+Y/+Z, Iso~1/2):} Abaqus-style camera presets.
    \item \textbf{Clipping mask:} Automatically hides the removed material region using the plane settings loaded from \texttt{data/e\{n\}\_plane\_settings.json}.
    \item \textbf{Deformation scale:} Amplifies displacement field for visual clarity.
    \item \textbf{Export Image} / \textbf{Export Data:} Save a screenshot or the current field array as PNG or CSV/NPY.
\end{itemize}

\subsection{Option B: ParaView}

\begin{enumerate}
    \item Open \textbf{ParaView}.
    \item \textbf{File $\rightarrow$ Open} $\rightarrow$ navigate to the output folder.
    \item Select a \texttt{.xdmf} file and click \textbf{Apply}.
    \item In the toolbar, change the variable to \texttt{stress\_tensor} and use \textbf{Split Field} to select the $S_{33}$ component.
\end{enumerate}

\subsection{Interpretation Hints}
\begin{itemize}
    \item \textbf{Red zones (positive):} Tensile residual stress.
    \item \textbf{Blue zones (negative):} Compressive residual stress.
    \item \textbf{Validation:} $S_{33}$ should be near zero at the cut face (boundary condition enforcement).
    \item \textbf{Clipping mask off:} Disable the mask in the Visualization tab to inspect the full geometry including the removed region.
\end{itemize}