\section{Step 1: Simulation Configuration}
\label{sec:guide_config}

\textbf{Goal:} Generate the master \texttt{config.json} file that controls all simulation parameters.

\subsection{Action: Using the GUI}
Navigate to the experiment data folder and launch the interface:

\begin{lstlisting}[language=bash]
cd src/Exp_Data/s1_exp
python write_input_gui.py
\end{lstlisting}

\subsection{Configuration Checklist}
Fill in the fields as follows:

\begin{enumerate}
    \item \textbf{Material Properties:}
    \begin{itemize}
        \item \textit{Elastic Modulus:} e.g., 210000 (MPa)
        \item \textit{Poisson Ratio:} e.g., 0.3
    \end{itemize}

    \item \textbf{Design of Experiments (DOE):}
    Define the search space for the sensitivity analysis.
    \begin{itemize}
        \item \textit{Mesh Range:} Min=0.6, Max=0.8, Step=0.1
        \item \textit{Length Range:} Min=50, Max=100, Step=25
    \end{itemize}
    \textit{Note:} The software creates a job for every combination (e.g., Mesh0.6-Len50, Mesh0.6-Len75...).
    \newline
    \textit{Note:} If you put something like  Min=0.6, Max=0.6, Step=0.1, the software will break.

    \item \textbf{Validation \& Save:}
    \begin{itemize}
        \item Click \textbf{"Save JSON"}: Confirm the success message.
    \end{itemize}
\end{enumerate}

\subsection{Verification}
Go to \texttt{src/Exp\_Data/s1\_exp/config/} and check if \texttt{config.json} was created. Open it with a text editor to ensure the paths (e.g., \texttt{"work\_directory"}) match your machine's structure.