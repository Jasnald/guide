\section{Step 1: Process Experimental Data}
\label{sec:guide_config}

\textbf{Goal:} Pre-process raw measurement files and compute polynomial surface/curve models that the simulation scripts will consume.

\subsection{Experiment 1 --- 3-D Surface (T-profile)}

\begin{lstlisting}[language=bash]
python scripts/e1_process_run.py
\end{lstlisting}

Reads all measurement files from \texttt{data/input/exp1/}, applies the side-mirroring rule for \texttt{Side2}, fits a degree-4 2-D polynomial surface, and writes JSON results to \texttt{data/output/exp1/surface\_data/}.

\subsection{Experiment 2 --- 1-D Curves (Milled Block)}

\begin{lstlisting}[language=bash]
python scripts/e2_process_run.py
\end{lstlisting}

Reads all curve measurement files from \texttt{data/input/exp2/}, applies mirroring for side \texttt{"2"}, fits a degree-2 polynomial with Ridge regularisation (\texttt{alpha=1.0}), and writes JSON results to \texttt{data/output/exp2/curve\_data/}.

\subsection{Verification}
After either script finishes, confirm that \texttt{*.json} files (including \texttt{Average.json} for Exp1) are present in the respective output directory. Use the notebooks in \texttt{notebook/} to visually inspect the fits before proceeding.

\section{Step 2: Configure Cutting-Plane Geometry}
\label{sec:guide_geometry}

\textbf{Goal:} Define the ZX and ZY cutting-plane positions and the material-removal region. These values are saved to \texttt{data/e\{n\}\_plane\_settings.json} and consumed by the CAE scripts.

\subsection{Action: Using the GUI}

\begin{lstlisting}[language=bash]
# Experiment 1
python scripts/setup_geometry.py --e1

# Experiment 2
python scripts/setup_geometry.py --e2

# Interactive selection
python scripts/setup_geometry.py
\end{lstlisting}

The \texttt{plane\_selector} window opens (\texttt{customtkinter}). The specimen outline is drawn automatically from the sample geometry file.

\subsection{Configuration}
\begin{enumerate}
    \item \textbf{Plane ZX (Horizontal Cut):} Enter the Y position in mm. The horizontal dashed line updates in real time.
    \item \textbf{Plane ZY (Vertical Cut):} Enter the X position in mm. The vertical dash-dot line updates.
    \item \textbf{Removal Region:} Enter the reference point (X, Y) inside the region to be removed. The hatched rectangle shows which side will be cut away.
    \item Click \textbf{SAVE JSON}: the file is written and the window closes automatically.
\end{enumerate}

\subsection{Verification}
Open \texttt{data/e1\_plane\_settings.json} (or \texttt{e2\_plane\_settings.json}) in a text editor and confirm that \texttt{plane\_zx}, \texttt{plane\_zy}, and \texttt{remove\_region} contain the expected coordinates.